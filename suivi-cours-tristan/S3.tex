\documentclass[document.tex]{subfiles}

\begin{document}

\section{S3}

\subsection{Contenue}

\begin{enumerate}
\item Rappel cours 2, exercices série 2
\item Modèle échantillonné : application de la matrice de transition
\item Retour d’état par placement de pôles
\item Commandabilité
\item Forme commandable  (tf $\Rightarrow$ ss)
\end{enumerate}

\subsection{Rappels cours 2}
\begin{itemize}
\item Valeurs propres, vecteurs propres de la matrice système A
\item Diagonalisation de la matrice A
\item Forme modale
\item Solution générale de la trajectoire x[k], temps discret
\item Solution générale de la trajectoire x(t), temps continu
\item Matrice de transition, exponentielle matricielle
\end{itemize}


\subsubsection{Solution générale de la trajectoire x(t) temps continu }

\figc{1}{50}

L'exponentiel matricielle $e^{A t}$ est appelée « matrice de transition » Matlab : expm(A*t)

\subsection{Définition de l'exponentiel matricielle : Taylor}

\figc{0.5}{51}

Les puissances $(A t)^2$, $(A t)^3$, etc sont à calculer selon \textbf{les règles de multiplication matricielle, pas élément par élément !}

\subsection{Exponentielle matricielle : unification}

\figc{0.9}{52}

\subsection{Modèle échantillonné du système à régler}

\figc{0.9}{53}

Trouver $H(z)$ à partir de $G_a(s)$ :trouver la représentation dans l'espace d'état de $H$ à partir de celle de $G_a$.

\subsection{Calcul de H(z) basé sur la réponse impulsionnelle}

\figc{0.9}{54}

\subsection{Calcul du modèle échantillonné dans l'espace d'état}

Si l'on souhaite utiliser l'espace d'état dans le domaine numérique (modèle échantillonné) les matrices A et B doivent être adaptées à l'aide des relations :

\figc{0.9}{55}

\subsection{Calcul par Matlab}

\figc{0.8}{56}

\subsection{Retour d'état  « state feedback »}

\textbf{les \textbf{trois objectifs} de la régulation :}
\begin{enumerate}
\item Rendre \textbf{la boucle fermée stable} avec une « dynamique » appropriée
\item Bonne régulation de \textbf{correspondance}, le signal à régler devrait au mieux suivre la consigne
\item Bonne régulation de \textbf{maintien}; maintenir le signal à régler proche de la consigne
\end{enumerate}

Hypothèse : une mesure complète du vecteur d'état est disponible.\\
Idée : reboucler l'état x(t) sur l'entrée u(t) avec un gain K.\\

\textbf{Aussi dit : "\textit{Full Information}"}
\figc{1}{72}

\subsection{Stabilité en B.F. - Méthode de synthèse du retour d'état par placement de pôles}

\figc{0.8}{57}

\subsubsection{Méthode}
\begin{enumerate}
	\item On s'intéresse uniquement au système à régler : $\boxed{\dot{x}= x + Bu}$
	\item On cherche un régulateur proportionnel dont le gain $K$ rend le système stable : $\boxed{u=-Kx}$
	\item On pose la relation en boucle fermée B.F. : $\boxed{\dot{x}=\underbrace{A-BK}x_{Abf}}$
	\item On trouve le polynôme caractéristique : $\boxed{\text{det}(SI-A) = 0}$
	\item On fixe des pôles pour le système régulé
	\item On trouve les valeurs de $K$ à l'aide des pôles placé précédemment
\end{enumerate}

\subsubsection{Exemple}

\figc{0.8}{58}

\figc{0.8}{59}

\figc{0.8}{60}

\subsection{Commandabilité}

\figc{0.8}{61}

\subsection{Critère algébrique pour la commandabilité}

\figc{0.8}{62}

\figc{0.8}{63}

\figc{0.8}{64}

\subsection{Exemples intuitifs}

\figc{0.8}{65}

Dessiner le schéma bloc associé : on voit tout de suite qu'il y a un sous-système qui est dans l'air !

\figc{0.8}{73}

\subsubsection{Autre exemple :}

\figc{0.8}{66}

\subsection{Exemple moins trivial}

\figc{0.8}{67}

\textbf{Exemple de calcul de la matrice de commandabilité : }\\
\begin{equation}
\begin{array}{c c}
A = \begin{bmatrix} -1 & 0 \\ 1 & -2 \end{bmatrix} & B = \begin{bmatrix} 1 \\ 1 \end{bmatrix} \\[12pt]
AB = \begin{bmatrix} -1 \cdot 1+0 \cdot 1 \\ 1 \cdot 1 - 2 \cdot 1 \end{bmatrix} = \begin{bmatrix} -1 \\ -1 \end{bmatrix} & \boxed{n=2} \\[12pt]
P_c = \big[ B \; \vline \;  AB \big] = \begin{bmatrix} 1 & -1 \\ 1 & -1 \end{bmatrix}
\end{array}
\end{equation}

\subsection{La forme commandable (tf $\Rightarrow$ ss)}

\begin{center}
\begin{tabular}{c c c}
\textbf{Temporel} & & \textbf{Fréquentiel} \\[12pt]
\codeword{ss} & $\Rightarrow$ & \codeword{tf} \\[12pt]
$A,B,C,D$ & $\Leftarrow $& $G(s)=C(SI - A)^{-1}B + D$ \\[12pt]
\centered{$\bullet$ \textcolor{red}{Forme modale} $\text{diag}(A)$\\
		  $\bullet$ \textcolor{red}{Forme commandable}}
\end{tabular}
\end{center}

\figc{0.8}{68}

\figc{0.8}{69}

\subsubsection{Schéma bloc de la forme commandable}

\figc{0.8}{70}

\figc{0.8}{71}

\begin{center}
\textbf{\underline{FORME COMMANDABLE}}
\end{center}


\begin{equation}
\boxed{
\begin{array}{c c}
A = \begin{bmatrix} 0 & \textcolor{mygreen}{1} & 0 \\ 0 & 0 & \textcolor{mygreen}{1} \\
	\textcolor{mygreen}{-a_0} & \textcolor{mygreen}{-a_1}  & \textcolor{mygreen}{-a_2} \end{bmatrix} &
	B = \begin{bmatrix} 0 \\ 0 \\ \textcolor{mygreen}{1}	\end{bmatrix} \\[24pt]
	C = \begin{bmatrix} \textcolor{mygreen}{b_0} & \textcolor{mygreen}{b_1}  & \textcolor{mygreen}{b_2} 	\end{bmatrix} & D = 0 \\[12pt]
	\Downarrow \\
	\boxed{G(s) = \frac{b_2 s^2 + b_1 s + b_0}{s^3 + a_2 s^3 + a_1 s + a_0}} & \boxed{\centered{\text{Degré relatif : } \\ d = n-m = 3-2 = 1 = B}}
\end{array}}
\end{equation}


\end{document}
