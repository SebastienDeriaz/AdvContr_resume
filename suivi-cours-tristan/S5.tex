\documentclass[document.tex]{subfiles}

\begin{document}

\section{S5}

\subsection{Contenue}

\figc{1}{100}

\subsection{Rappel}

\subsubsection{LQR (Linear Quadratic Regulator)}

\figc{1}{101}

\begin{tabular}{l c l}
$Q \geq 0$ & : & Semi-défini positif \\
$R > 0$ & : & défini positif
\end{tabular}

\subsubsection{Solution LQR analogique / équation de Riccati}

\figc{1}{102}
\subsubsection{Cheap control / Expensive control}

On souhaite régler un système avec une minimum d'énergie.\\

\textbf{Il y a deux cas de figure : }
\begin{enumerate}
\item Le cas extrême $R = 0$ est appelé \textit{« \underline{cheap control} »}
Le gain du retour d'état va tendre vers \textbf{infini} !
\item L'autre c\textit{as extrême $Q = 0$ est appelé }« underline{expensive control} »
Les pôles déjà stables en boucle ouverte restent inchangés en b.f.
Un pôle instable en b.o. miroite autour de l'axe imaginaire. \\
\end{enumerate}

\subsubsection{DLQR (Discrete Linear Quadratic Regulator)}

\figc{1}{103}

\subsubsection{Solution LQR discret / Équation de Riccati}

\figc{1}{104}

\subsection{Nomenclature anglo-saxonne}

\figc{1}{105}

\begin{tabular}{l c c c l}

$P(s)$ : \textit{plant} & $=$ & système à régler & $=$ & $G_a(s)$ \\

$C(s)$ : \textit{controller} & $=$ & s régulateur & $=$ & $G_c(s)$ \\

$L(s)$ : \textit{open loop} & $=$ & boucle ouverte & $=$ & $G_o(s)$ \\

$r(t)$ : \textit{reference signal} & $=$ & consigne & $=$ & $w(t)$ \\

$T(s)$ : \textit{closed loop} & $=$ & boucle fermée & $=$ & $G_{yw}(s)$ \\

\end{tabular}

\subsection{Rappel : boucle ouverte et boucle fermée}

\figc{1}{106}

\subsubsection{Boucle ouverte pour le retour d'état}

\figc{1}{107}

$$\boxed{L(s)=K(sI-A)^{-1}B}$$

\subsection{Critère de Nyquist simplifié}

\figc{1}{108}

\subsubsection{Réflexions intuitives pour motiver le critère de Nyquist}

\figc{1}{109}

\subsubsection{Effet Larsen : feedback acoustique provoquant un sifflement}

\figc{1}{110}

\subsubsection{Réflexions intuitives}

\figc{1}{111}

\subsubsection{Explication mathématique}

\begin{equation}
\begin{array}{c}
boxed{G_{yw}(s)=\frac{G_o(s)}{1+G_o(s)}} \\
boxed{G_o{s}=-1 \; \Rightarrow \; G_{yw}(s) = \frac{-1}{1-1} = \frac{-1}{0} \rightarrow \infty}
\end{array}
\end{equation}


\subsubsection{Boucle fermée marginalement stable $\rightarrow$ un sinus peut persister dans la boucle !}

\figc{0.8}{112}

\subsubsection{Boucle fermée marginalement stable}

\figc{0.8}{113}

\subsubsection{Réflexions intuitives}
Pour avoir un amortissement en boucle fermée (stabilité), il faut que le signal contre-réactionné y soit \textbf{plus petit} que le signal e injecté dans la boucle.

\subsubsection{Définition : marge de gain et de phase}

\figc{1}{114}

\subsubsection{Critère de Nyquist simplifié}
 $\phi_m$ doit être positive en dB et $A_m$ doit être positif en dB.\\

\subsubsection{Lieu de Nyquist - Alternative au diagramme de Bode}

Au lieu des 2 plots séparés du diagramme de Bode (amplitude et phase), ayant comme abscisse la pulsation, le \textbf{lieu de Nyquist} est \textbf{un seul} plot.\\

Le lieu de Nyquist est la courbe paramétrique de $G_o(j\omega)$ dans le plan complexe.\\

Pour chaque valeur de la pulsation $\omega$ cela donne un point $G_o(j\omega)$ dans le plan complexe.\\

En parcourant $\omega$ de zéro à $\infty$ la valeur de $G_o(j\omega)$ varie, et cela donne une courbe paramétrique orientée, appelée de le "lieu de Nyquist".

\subsubsection{Critère de Nyquist simplifié}

\figc{1}{115}

La boucle fermée est stable si \textbf{le lieu de Nyquist \underline{n'encercle pas} le point critique}. Donc si \textbf{le point critique est à \underline{gauche} du lieu de Nyquist}. Que valable si la boucle ouverte est stable !

\subsubsection{Motivation pour le critère de Nyquist généralisé}

\figc{1}{116}

\textbf{Calculons le pôle en boucle fermée pour déterminer la stabilité en b.f : }

\figc{1}{117}

Dans cette exemple $s$ : est le pôle en boucle fermé du système.

\subsubsection{Lieu de Nyquist complet de $\frac{k}{s-1}$}

\figc{1}{118}

\textbf{On cherche à trouver la valeur de $K$ qui permet que le lieu de Nyquist encercle le point critique : }
\figc{1}{119}

\figc{1}{120}

\textbf{\underline{Attention} : Cette conclusion n'est valable que pour cette exemple particulier.}

\subsection{Critère de Nyquist}

\subsubsection{Critère de Nyquist simplifié (critère du revers) :}
$\rightarrow$ \; Valable stables ou instables en b.o.\\ 
Critère basé sur le nombre d'encerclements de $G_o(j \omega)$ autour du point critique $-1$ pour les systèmes stables ou instables en b.o.\\ 
Critère basé sur le nombre d'encerclements de $G_o(j \omega)$ autour du point critique $-1$

\subsubsection{Critère de Nyquist généralisé :}

$\rightarrow$ \; valable pour tous les systèmes \textbf{stables ou instables en b.o.\\ 
Critère basé sur le nombre d'encerclements de $G_o(j \omega)$ autour du point critique $-1$}.\\

\textbf{La \underline{boucle fermée} est stable si et seulement si le \underline{nombre d'encerclements du point critique -1} dans le sens trigonométrique correspond au \underline{nombre de pôles instables} en \underline{boucle ouverte}.}

\subsection{critère de Nyquist généralisé}

La boucle \textbf{fermée} est stable, si et seulement si le nombre \textbf{d'encerclements} du lieu de Nyquist complet de la \textbf{boucle ouverte} $L(j\omega)$, $-\infty < \omega < \infty $ autour du point critique $s=−1$ correspond \textbf{exactement} aux nombre de pôles \textbf{instables} $N_p$ en boucle ouverte. Les encerclements sont comptés dans le sens trigonométrique.

\figc{1}{121}

\subsubsection{Vérification des pôles en boucle fermée}

\figc{1}{122}

\subsubsection{Rappel cours AAV}
\figc{1}{123}

\figc{1}{124}

\subsection{Régulateur LQR}

\subsubsection{Robustesse du régulateur LQR (retour d'état)}

\textbf{\underline{Inégalité de Kalman}} : $\mid 1+L(j \omega) \mid \geq 1$ pour chaque pulsation $\omega \; \Leftrightarrow$ distance entre le lieu de Nyquist $L(j \omega)$ et le point critique $\mid 1+L(j \omega) \mid \geq 1$ $\Leftrightarrow$ marge de phase $\phi_m \geq 60\degres$ et marge de gain $0.5\leq Am < \infty$.\\

\figc{0.8}{125}

\subsubsection{Régulation en correspondance LQR avec action intégrale}

\figc{1}{132}

\begin{enumerate}
\item Faire d'abord la synthèse LQR de $K_2$ (vecteur ligne), puis après celle de $K_1$ (scalaire).\\
La boucle extérieure devrait être plus lente que la boucle intérieure.\\
\textbf{\underline{Inconvénient :}} la synthèse de $K_1$ peut compromettre la synthèse LQR de $K_2$
\item \textbf{\underline{Mieux}} : Faire une seule synthèse simultanée pour $K_1$ et $K_2$ !\\
\textbf{\underline{Idée principale :}} l'action intégrale rajoute une variable d'état, ce qui agrandit la taille de $A$ et de $B$
\end{enumerate}

\begin{center}
$2=y-r$ \textbf{Attention au \underline{signe} !}
\end{center}

Sa dérivée temporelle donne :

$$\dot{e}=\dot{y}-\overbrace{\dot{r}^0}=\dot{y=C\dot{x}}$$

Définir deux variables intermédiaires, $z$ et $w$ , comme :
$$ z=\dot{x} \text{ et } w = \dot{u}$$

un nouveau système en résulte :
$$ \overbrace{\begin{bmatrix} \dot{e} \\ \dot{z} \end{bmatrix}}^{\dot{\hat{x}}} = \overbrace{\begin{bmatrix} 0 & C \\ 0 & A \end{bmatrix}}^{\dot{\hat{A}}} \overbrace{\begin{bmatrix} e \\ z \end{bmatrix}}^{\hat{x}} + \overbrace{\begin{bmatrix} 0 \\ B \end{bmatrix}}^{\hat{B}} \overbrace{w}^{\hat{u}}$$

\figc{1}{134}

\subsection{Observateur}

\subsubsection{Motivation observateur}
\figc{1}{135}
\figc{1}{136}

\subsubsection{Observateur pour monitorer des signaux difficilement mesurables}
\figc{1}{137}

\subsubsection{Idée naïve : observateur "trivial"}
\figc{1}{138}

\subsubsection{Dynamique de l'erreur de l'observateur trivial}
\figc{1}{139}

\begin{itemize}
\item La dynamique de l'erreur est indépendante du signal de commande u(t).
\item $x_e(t)=e^{At}x_{e,0}$ : sous quelle condition l'observateur trivial converge ?
\item Converge que si le processus est stable, i.e. si toutes les valeurs propres ont une partie réelle négative
\end{itemize}

\subsubsection{Points faibles de l'observateur trivial}
\figc{1}{140}

\subsubsection{Structure de l'observateur complet}
\figc{1}{141}


\subsubsection{Dynamique de l'erreur de l'observateur complet}
\figc{1}{142}
\figc{1}{143}












\subsubsection{Rappel cours AAV - Design d'un régulateur robuste par loopshaping manuel}

\figc{1}{126}
\figc{1}{127}
\figc{1}{128}
\figc{1}{129}
\figc{1}{131}





\end{document}

