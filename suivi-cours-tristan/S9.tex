\documentclass[document.tex]{subfiles}

\begin{document}

\section{S9}

\subsection{Contenu}

\begin{enumerate}
\item Rappel cours 8 : décomposition SVD, exercices série 8
\item Systèmes non-linéaires
\begin{itemize}
\item points d'équilibre (points de fonctionnement)
\item linéarisation autour d'un point d'équilibre (Taylor multi-variable)
\item redressement (compensation) de la non-linéarité par feedback
\item effets spéciaux de systèmes non-linéaires
\end{itemize}
\end{enumerate}

\subsection{Rappel cours 8}

\subsubsection{Image du cercle unité (hypersphère unité)}

\figc{1}{201}

\subsubsection{Amplification maximale et minimale}

Définition équivalente de l'amplification maximale $=\sigma_{max}$
\figc{0.5}{202}

Définition de l'amplification minimale $=\sigma_{min}$
\figc{0.5}{203}

\subsection{Factorisation (décomposition) SVD de matrices}

\textbf{SVD (Singular value decomposition)} : \url{https://en.wikipedia.org/wiki/Singular_value_decomposition}

\textbf{SVD (Décomposition en valeurs singulières)} : \url{https://fr.wikipedia.org/wiki/D%C3%A9composition_en_valeurs_singuli%C3%A8res}

\figc{1}{204}

\subsubsection{décomposition en valeurs singulières (SVD)}
\figc{1}{205}
\textbf{Réécriture de la relation}

$$ \boxed{M = U S V^T} $$

$$ M = \underset{\underset{mtx ortho}{\downarrow}}{U} \underset{\underset{mtx diag}{\downarrow}}{S} \underset{\underset{mtx ortho transp}{\downarrow}}{V^T} $$

avec : $U = \begin{bmatrix} \vec{u}_1 & \vec{u}_2 & \ldots & \vec{u}_n \end{bmatrix} $

$$ M \vec{x}  = \sigma_1 \overbrace{vec{u}_1 \underbrace{\vec{v}_2^T \vec{x}}_{\text{scalair}}}^{\text{mtx rang 1}} + \sigma_2 \overbrace{\vec{u}_2 \underbrace{\vec{v}_2^T \vec{x}}_{\text{scalair}}}^{\text{mtx rang 1}} + \ldots + \sigma_n \overbrace{\vec{u}_n \underbrace{\vec{v}_n^T \vec{x}}_{\text{scalair}}}^{\text{mtx rang 1}}$$

\subsubsection{amplification directionnelle}
\figc{1}{206}

\subsubsection{Rang d'une matrice, conditionnement}
\figc{1}{207}

\textbf{Exercice}
\figc{1}{208}

\subsubsection{Autre propriétés importantes de la SVD}

lien entre valeurs singulières et valeurs propres :\\
\begin{itemize}
\item valeurs singulières applicable pour n'importe quelle taille de matrice
\item valeurs propres que applicable pour des matrices de taille carrée
\item les valeurs singulières sont des nombres réels positifs $\geq 0$
\item les valeurs propres peuvent être réels (pos/nég) ou complexes 
\item $\sigma_k(M)=\sqrt{\lambda_k(M^T M)}$ \\
Ce qui explique que les valeurs singulières sont  $\geq 0$
\item $\text{det}(M)=\prod_{k=1 \ldots n} \sigma_k(M)$ pour une matrice carrée $M$\\
Ce qui explique que le déterminant peut être très loin de zéro alors que
\end{itemize}


\subsubsection{Diagramme de Bode multivariable (sigma plot)}

Pour une matrice de transfert $H(j\omega)$ d'un système multivariable (MIMO), les valeurs singulières de $G$ dépendent de la fréquence (respectivement la pulsation).\\

On les affiche en \si{[\decibel]} ou bien sur un axe vertical logarithmique.\\
L'axe horizontal est logarithmique, soit en \si{[\radian\per\sec]} ou \si{[\hertz]}.\\

Command Matlab : \codeword{sigma(sys)}\\

L'information de la phase est perdue.\\
L'information des directions d'amplification est perdue.

\figc{1}{209}

$$ G)s  = \begin{bmatrix} \frac{10^3}{s} & 0 \\ 0 & \frac{10}{1 + \frac{s}{10}} \end{bmatrix}$$

\textbf{Équivalent à :}

\begin{itemize}
\item Inversion :
$$ G)s  = \begin{bmatrix} \frac{10}{1 + \frac{s}{10}} & 0 \\ 0 & \frac{10^3}{s}  \end{bmatrix}$$
\item Déphasage :
$$ G)s  = \begin{bmatrix} \frac{10}{1 + \frac{s}{10}}e^{-T_{r1}s} & 0 \\ 0 & \frac{10^3}{s}e^{-T_{r2}s}  \end{bmatrix}$$
\item Changement de direction :
$$ G)s  = U(s)\begin{bmatrix} \frac{10}{1 + \frac{s}{10}}e^{-T_{r1}s} & 0 \\ 0 & \frac{10^3}{s}e^{-T_{r2}s}  \end{bmatrix} V^T(s)$$
Avec : $ U V^T = I $ et $U V^T = I $
\end{itemize}

Manière la plus simple de représenter un système multivariable, il y a cependant une perte d'information. En effet, on ne connais ni la phase, ni, la direction dans lequel il y a l'amplification maximal (il nous manque les vecteurs singulier).


\subsection{Points d'équilibre / points de fonctionnement}

On appelle $\vec{x}_e$ un \textbf{point d'équilibre} du système dynamique non-linéaire :

$$
\begin{array}{l l}
\dot{\vec{x}}=\vec{f}(\vec{x}) & \begin{cases} \dot{x}_1 = f_1(x_1,x_2,\ldots,x_n) \\ \dot{x}_2 = f_2(x_1,x_2,\ldots,x_n) \\
\vdots \\ \dot{x}_n = f_n(x_1,x_2,\ldots,x_n) \end{cases} \\
\text{si } \vec{f}(\vec{x}_e)=\vec{0}
\end{array}
$$

$$ \dot{\vec{x}}(t) = \vec{f}(\vec{x}(t),u(t)) \quad \quad \vec{f}(\vec{x}_e,u_e) = \vec{0}$$ 

Le \textbf{"point de fonctionnement"} dépend de l'entrée : $ \vec{x}_e(u_e)$

\subsection{Système non-linéaire d'ordre 1}
\figc{0.6}{210}

Un système non-linéaire peut avoir \textbf{plusieurs} points d'équilibres !\\

Exemple : pendule $\longrightarrow$ un point d'équilibre \textbf{stable en bas} et un point d'équilibre \textbf{instable en haut}.

\subsubsection{Le calcul d'un point d'équilibre}
Résolution d'un système "implicite" : $\boxed{\vec{f}(\vec{x}_e)=\vec{0}}$\\
Trouver le zéro de la fonction f(x) ! \\

Souvent pas de solution analytique !\\

Méthode numérique itérative de Newton-Raphson permet de trouver le zéro d'une fonction à partir d'une valeur de départ approximative ("initial guess")


\subsection{Approximation locale par Taylor (linéarisation)}
\figc{0.6}{212}

\tbox{La tangente approxime localement la fonction f(x)}


\subsubsection{Nouvelle variable d'état : la "déviation" autour du point d'équilibre}
%\figc{1}{213}

Rappel : $x_e$ correspond au point d'équilibre, donc la dérivé temporel de $x_e$ est nul !

$$ \begin{array}{l}
 \vec{\Delta x} = \vec{x} - \vec{x}_e \\
 \dot{\vec{\Delta x}} = \dot{\vec{x}} - \vec{0} = \dot{\vec{x}}
\end{array} $$

Taylor mono-variable :

$$ f(x_e + \Delta x) \approx f(x_e) + \dfrac{df}{dx} (x_e)\cdot \Delta x + \ldots $$

$$
\begin{array}{l}
\dot{x} = f(x) \\ \dot{\Delta x} \approx \underbrace{f(x_e)}_{0} + \underbrace{\dfrac{df}{dx} (x_e)}_{a} \Delta x
\end{array}
$$

Finalement le système devient linéaire :
$$ \boxed{\dot{\Delta x}=a\cdot \Delta x} $$

\subsubsection{Taylor multivariable}
\figc{0.4}{214}

$$ \vec{f}(\vec{x}_e + \vec{\Delta x}) = \vec{f}(\vec{x}_e) + \underbrace{\nabla \vec{f} \; \vert_{\vec{x}_e}}_{\text{mtx jcobienne}} \vec{\Delta x}$$

\subsection{Stabilité locale des points d'équilibres d'un système non-linéaire d'ordre n}
%\figc{0.4}{215}

$$ \boxed{ \dot{\vec{\Delta x}} = A\vec{\Delta x}} $$

$$ A = \begin{bmatrix}
\frac{\partial f_1}{\partial x_1} & \frac{\partial f_1}{\partial x_2} & \ldots & \frac{\partial f_1}{\partial x_n} \\[6pt]
\frac{\partial f_2}{\partial x_2} & \frac{\partial f_2}{\partial x_2} & \ldots & \frac{\partial f_1}{\partial x_n} \\[6pt]
\vdots & \vdots & \ddots & \vdots \\[6pt] \frac{\partial f_n}{\partial x_1} & \frac{\partial f_n}{\partial x_2} & \ldots & \frac{\partial f_n}{\partial x_n} 
\end{bmatrix} $$

\subsection{Linéarisation autour d'un point d'équilibre}
%\figc{1}{216}

\begin{enumerate}
\item Si le système \textbf{linéarisé} est \textbf{asymptotiquement} stable (\textbf{toutes} les \textbf{valeurs propres de A} dans le\textbf{ demi-plan gauche}), alors le point d'équilibre $x_e$ du système \textbf{non-linéaire} est \textbf{\underline{localement}} \textbf{stable}.
\item Si le système \textbf{linéarisé} est \textbf{instable} (\textbf{une} ou \textbf{plusieurs valeurs propres de A} dans le \textbf{demi-plan droit}), alors le point d'équilibre $x_e$ du système non-linéaire est également \textbf{instable}.
\item Si le système \textbf{linéarisé} est \textbf{marginalement stable} (valeurs propres de A sur l'axe imaginaire), alors on ne peut \textbf{rien conclure} sur le comportement du système non-linéaire.

\end{enumerate}

\subsubsection{Récapitulatif des système non-linéaire :}
\textbf{Système non-linéaire :} \hfill $ \boxed{\dot{\vec{x}}=\vec{f}(\vec{x})}$ \hfill \; \\

Linéarisation autour d'un point de fonctionnement (point d'équilibre) :

$$ \vec{f}(\vec{x}_e) = \vec{0} $$

Système linéarisé  : $\Delta \vec{x} = \vec{x} - \vec{x}_e$
$$ \boxed{\dot{\Delta \vec{x}} = A \Delta \vec{x}} $$

\begin{enumerate}
\item \textbf{Système linéarisé \underline{stable}} $\Rightarrow$ point d'équilibre du système non-linéaire est \textbf{\underline{localement} stable}
\item \textbf{Système linéarisé \underline{instable}} $\Rightarrow$ \textbf{point d'équilibre} du système non-linéaire est \textbf{\underline{localement} instable}
\item \textbf{Système linéarisé \underline{marginalement stable}} $\Rightarrow$ \textbf{\underline{aucune} conclusion possible}
\end{enumerate}

\subsection{Linéarisation d'un système non-linéaire avec entrée}
%\figc{0.6}{217}

$$
\begin{array}{l l}

\dot{\vec{x}} = \vec{f}(\vec{x},u) & \vec{f}(\vec{x}_e,u_e)=\vec{0} \\

A = \dfrac{\partial \vec{f}}{\partial \vec{x}}\bigg\rvert_{\vec{x}_e,u_e} = \begin{bmatrix}
\frac{\partial f_1}{\partial x_1} & \frac{\partial f_1}{\partial x_2} & \ldots & \frac{\partial f_1}{\partial x_n} \\[6pt]
\vdots & \vdots & \vdots & \vdots \\[6pt]
\frac{\partial f_n}{\partial x_1} & \frac{\partial f_n}{\partial x_2} & \ldots & \frac{\partial f_n}{\partial x_n} \\[6pt]
\end{bmatrix} \Bigg\rvert_{\vec{x}_e,u_e} &
 B = \dfrac{\partial \vec{f}}{\partial \vec{u}}\bigg\rvert_{\vec{x}_e,u_e} = \begin{bmatrix}
\frac{\partial f_1}{\partial u} \\[6pt]
\frac{\partial f_2}{\partial u} \\[6pt]
\vdots \\[6pt]
\frac{\partial f_n}{\partial u} \\[6pt]
\end{bmatrix}
\end{array}
$$
$$\boxed{ \dot{\vec{\Delta x}} = A \Delta x + B \Delta u}$$

\subsection{Synthèse d'un régulateur basé sur un modèle linéarisé}
%\figc{1}{218}

Que faire si le point de fonctionnement (point d'équilibre) varie ?\\

S'il varie lentement, on peut changer les paramètres du régulateur en fonction du point de fonctionnement "gain scheduling" !\\

Si le système non-linéaire est bien connu, on peut aussi essayer de "redresser" les non-linéarités. Approche en deux étapes :
\begin{enumerate}
\item En utilisant un \textit{feedback/feedforward} non-linéaire, rendre le système linéaire. On essaye de compenser les non-linéarités
\item On obtient un nouveau système à régler qui est linéaire.\\
Méthodes de synthèse linéaire  


\subsection{Linéarisation autour d'une trajectoire périodique}

\figc{0.8}{219}

$$\boxed{\dot\vec{x}(t) = A(t) \vec{\Delta x} (t)}$$

Le système linéarisé n'est plus LTI (linear time invariant), mais LTV (linear time variant), resp. LTP (linear time periodic).

\subsection{Redressement de la non-linéarité : exemple}

\figc{0.8}{220}
\figc{0.8}{221}

\subsubsection{Redressement de la nonl-inéarité : "feedback linearization"}
\figc{0.8}{222}

\subsection{Effets observés dans les systèmes non-linéaires}
\figc{0.8}{223}
\figc{0.8}{224}
\figc{0.8}{225}


















\end{enumerate}
















\end{document}

