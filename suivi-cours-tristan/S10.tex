\documentclass[document.tex]{subfiles}

\begin{document}

\section{S10 - Système non-linéaire - Identification}

\subsection{Contenu}
\begin{enumerate}
\item Rappel cours 9, exercices série 9
\item Modèle échantillonné dans l'espace d'état avec un retard non multiple entier de h 
\item Identification
\end{enumerate}

\subsection{Rappel Cours S9}

\subsubsection{Linéarisation d'un système non-linéaire avec entrée}

\textbf{Rappel : }\\
Un point d'équilibre signifie que la dérivé $\dot{x}$ qui correspond au point de fonctionnement $\vec{f}(\vec{x}_e,u_e)$ est nul. Soit : $\boxed{\vec{f}(\vec{x}_e,u_e) = 0}$\\

$$
\begin{array}{c c c}
\text{Système non-linéaire} & \quad \quad \quad \quad \quad \quad  & \text{Sytème linéarisé}\\
\dot{\vec{x}}=\vec{f}(\vec{x},u) & \longrightarrow & \dot{\overrightarrow{\Delta x}} = A \Delta x + B \Delta u
\end{array}
$$

\textbf{Méthode en résumé}
\begin{enumerate}
\item Trouver les points d'équilibre du système : $\boxed{\vec{f}(\vec{x}_e,\vec{u}_e)= \vec{0}}$ 
\item Calculer les Jacobien du système : 
$$
\boxed{A = \dfrac{\partial \vec{F}}{\partial \vec{x}} = J_{A} = \begin{bmatrix} \frac{\partial F_1}{\partial x_1} & \frac{\partial F_1}{\partial x_2} & \ldots & \frac{\partial F_1}{\partial x_n}\\ \vdots & \vdots & \ldots & \vdots \\ \frac{\partial F_n}{\partial x_1} & \frac{\partial F_n}{\partial x_2} & \ldots & \frac{\partial F_n}{\partial x_n} \end{bmatrix}} \quad \quad \quad \boxed{B = \dfrac{\partial \vec{F}}{\partial u} = J_{B}= \begin{bmatrix} \frac{\partial F_1}{\partial u} \\  \frac{\partial F_2}{\partial u} \\ \vdots \\ \frac{\partial F_n}{\partial u} \end{bmatrix}} $$
\item Linéariser le système au autours des points d'équilibre en évaluant le Jacobien du système au point d'équilibre $x_{e1},x_{e2},\ldots$ 
$$
\boxed{ J\big\vert_{x_{en}}= \begin{bmatrix} \frac{\partial F_1}{\partial x_1} & \frac{\partial F_1}{\partial x_2} \\ \frac{\partial F_2}{\partial x_1} & \frac{\partial F_2}{\partial x_2}  \end{bmatrix} \Bigg\vert_{x_{en}}}
$$
\item Le système linéaire trouvé est décrit par :
$$
\boxed{\dot{\overrightarrow{\Delta x}} = A \Delta x + B \Delta u}
$$
\item Est-ce que le point d'équilibre évalué $x_{ek}$ est localement stable ?\\
\textbf{Rappel : } Si le \textbf{système linéarisé} est \textbf{globalement stable} $\rightarrow$ Le \textbf{système non-linéaire} est \textbf{localement stable} au point d'équilibre évalué.\\
Le système linéarisé est stable si ces valeurs propres $\lambda_k$ se trouvent dans le demi plan gauche $\Rightarrow$ $\boxed{\text{Re}(\lambda_k)<0}$.
\item \textbf{Astuce} : Si la matrice du système linéarisé est triangulaire $\longrightarrow$ les valeurs propres correspondent à la diagonal de la matrice :
$$
\lambda_k \Rightarrow det(\lambda I-M) = 0 \Rightarrow det(\begin{bmatrix} \lambda - a & 0 & 0 \\ b & \lambda - c & 0 \\ d & e & \lambda - f \end{bmatrix} = 0 \Rightarrow \begin{cases} \lambda_1 = a \\ \lambda_2 = c \\ \lambda_3 = f \end{cases}
$$
\end{enumerate}

\figc{0.8}{226}

\subsubsection{Redressement de la non-linéarité : "feedback linearization"}

\figc{1}{227}

\subsubsection{Effets observés dans les systèmes non-linéaires}

\figc{1}{228}

\figc{1}{229}

On peut observer le phénomène de cycles limites dans les systèmes numériques. En effet les convertisseur AD/DA sont non linéaire et un peut observer un comportement de cycles limites sur les bits de poids faible.

\subsection{Calcul de $H(z)$ basé sur la réponse impulsionnelle}

\figc{1}{230}

\subsection{Propriétés du "modèle échantillonné"}

\begin{enumerate}
\item Les pôles analogiques $𝑠$ se transforment selon $z= e^(h s)$\\
Il n'y a pas de formule pour la transformation des zéros.
\item Le gain statique est préservé : $H(z=1)  =  G_a (s=0)$
\item Le modèle échantillonné $H(z)$ est linéaire en $G_a(s)$ \\
Une somme $G_{a1} (s) + G_{a2}(s)$ donne $H_1(z) + H_"(z)$\\
\textbf{Attention :} Pas valable pour un produit !
\item Un retard pur d'un multiple entier $N$ de périodes d'échantillonnages côté analogique, $e^{-N h}$ donne lieu à un facteur $z^{-N}$ dans le modèle échantillonné.
\end{enumerate}

\textbf{Retard pur analogique : } Difficile à traiter $\Rightarrow$ $\boxed{e^{-T_r \cdot j \omega}=e^{-T_r \cdot s}}$
$$
e^{-T_r \cdot s} \underbrace{\approx}_{\text{approx Padé}} \dfrac{1-\frac{T_r}{2}\cdot s}{1-\frac{T_r}{2}\cdot s}
$$

$$
	\vert e^{-j\omega T_r} \vert = 1 \quad \quad \quad \vert e^{j \phi} \vert = 1
$$

$$
\phi = arg(e^{-T_r \cdot j \omega}) = -T_r \cdot \omega
$$


\subsection{Calcul du modèle échantillonné dans l'espace d'état avec un retard non multiple entier de la période d'échantillonnage}


\figc{1}{231}

On convolu le signal d'entrée $u$ par la condition initiale.

\figc{1}{232}

\subsubsection{Sans retard}

$$
\begin{cases}
x[k+1] = A_n x[k] + B_n u[k]
y[k] = c_n x[k]
\end{cases}
$$

$$
\begin{array}{l}
\boxed{A_n = e^{A h}} \quad \quad \quad \boxed{B_n = \int^h_0 e^{A \tau} B d\tau} \quad \quad  \quad \boxed{C_n = C}
\end{array}
$$



















\end{document}
