\documentclass[document.tex]{subfiles}

\begin{document}

\section{S7 - Programme Advanced Control, cours 7}


\subsection{Contenue}

\figc{1}{166}

\subsection{Rappel}

\subsubsection{Structure de l'observateur}

\figc{1}{167}

\subsubsection{Dynamique de l'erreur de l'observateur complet}

\figc{1}{168}

\subsubsection{Critère de l'observabilité}

\figc{1}{169}

Un système est observable, s'il est toujours possible de calculer la condition initiale x[0] du vecteur d'état à partir des mesures y[0], y[1], ..., y[n-1] du signal de sortie sur une durée de n coups d'horloge.\\

\subsubsection{Observabilité}

\figc{1}{170}


Un système est \textbf{observable} si le \textbf{rang} de la \textbf{matrice d'observabilité} $P_o$ est n\\
En monovariable (SISO) : si $det(P_o)\neq 0$\\


\textbf{Observabilité :}\\
Où doit-on placer le \textbf{capteur} ("\textit{Observateur}")

\textbf{Commandabilité :}\\
Où doit-on placer l'\textbf{actionneur} ("\textit{Commande}")


\subsection{Application : palier magnétique}

Où placer les capteurs pour que les modes propres qui nous intéressent soient bien observables ?

\figc{1}{171}


\figc{1}{172}


\subsection{LTR : Loop Transfer Recovery}

\figc{1}{173}

LTR (\textit{Loop Transfer Recovery}) est une méthode de choix des pondérations $Q$ et $R$ de l'observateur, permettant de récupérer la bonne robustesse du retour d'état :
$$ \boxed{\phi_m > 60 \degres,\; A_m = [0.5 \; \ldots \; \infty]} $$

\begin{center}
\begin{tabular}{l l}
$Q = \rho B B^T$ et $R=1$ & $\rho \longrightarrow \infty$ : Observateur rapide
\end{tabular}
\end{center}


\textbf{\underline{Attention :} Ne fonctionne que pour les système à déphasage minimale !} \\

\textbf{Système à déphasage minimal :}\\
$\Leftrightarrow$ \textbf{Pas de zéro} dans le \textbf{demi-plan droite}.


\subsubsection{lieu de Nyquist en fonction de $\rho$}

\figc{1}{174}

\subsection{Application : observateur de vitesse}

\figc{1}{177}

\subsubsection{Problème de la dérivée numérique}

\figc{1}{178}

\figc{1}{179}

\subsubsection{Observateur de vitesse}

\figc{1}{180}

\subsubsection{Observateur de vitesse $=$ filtre complémentaire}

Après un petit calcul (Exercice 8.1) :\\

\figc{1}{181}

\begin{tabular}{|l c l r|}
\hline
$F_1(s)$ & : & Filtre passe haut & (gain statique : $F_1(0)=0$)\\
$F_2(s)$ & : & Filtre passe bas & (gain statique : $F_2(0)=1$) \\
$F_1(s) + F_2(s)$ & : & Filtre complémentaire & ($F_1(s)+F_2(s) = 1$)\\\hline
\end{tabular}

Pour la synthèse des gains $H_1$ et $H_2$, on peut p.ex. faire un placement de pôles, et par soucis de simplicité placer un double pôle réel négatif.\\

En jouant sur ce double-pôles, on modifie la pulsation de coupure des filtres complémentaires $F_1(s)$ et $F_2(s)$.\\

\textbf{Cas extrêmes :}
\begin{enumerate}
\item \textbf{Double pôle très négatif} $\Rightarrow$ pulsation de coupure très élevée $\Rightarrow$  gains $H_1$ et $H_2$ très élevés $\Rightarrow$  $F_1(s) = 0$, $F_2(s) = 1$. \\
On obtient le dérivateur de y(t) !
\item \textbf{Double pôle proche de $0$} $\Rightarrow$ pulsation de coupure très basse $\Rightarrow$ gains $H_1$ et $H_2$ très bas $\Rightarrow$ $F_1(s) = 1$, $F_2(s) = 0$. \\
On obtient l'intégrateur de u(t) !
\end{enumerate}

\subsection{Implémentation de régulateurs}
\subsubsection{Conseil d'implémentation (1/2)}

Aujourd'hui, environ $99\%$ des implémentations de régulateurs sont \textbf{numériques}.\\
Avec de rares \textbf{exceptions} : 
\begin{enumerate}
\item \textbf{Bande passante très élevée}, où même des solutions FPGA seraient trop lentes
\item \textbf{Exigences très faible bruit}
\end{enumerate}

\begin{itemize}
\item Choix de la période d'échantillonnage $t_s$ 
\item Problèmes liés à la numérisation dans le \textbf{temps}
\item Problèmes liés à la quantification des \textbf{signaux}\\
(No. bits des convertisseurs, word length, fixed point / floating point)
\item Problèmes liés à la quantification des \textbf{paramètres} \\
(word length, fixed point / floating point)
\end{itemize}

\subsubsection{Discrétisation dans le temps}

\textbf{Il existe deux approches possibles :}

\begin{enumerate}
\item [\textbf{A}] : D'abord faire une synthèse "\textbf{analogique}" c-à-d "temps continu", et après coup, \textbf{numériser} le \textbf{régulateur} avec la méthode "\textbf{Tustin}" = "\textit{méthode des trapèzes}"\\
$\rightarrow$ \textbf{Matlab} : \codeword{Gc\_num = c2d(Gc, h, 'tustin');} \\
Fonctionne bien si la fréquence d'échantillonnage est élevée par rapport à la dynamique du système
\item [\textbf{B}] : D'emblée travailler avec le modèle échantillonné du système à régler :\\ $\rightarrow$ \textbf{Matlab} :  \codeword{H = c2d(Ga, h, 'zoh');}\\
D'abord numériser le système à régler, ensuite faire la synthèse en numérique. En règle générale plus précis, donne des meilleurs résultats si la fréquence d'échantillonnage est faible.\\
Peut poser des problèmes numériques si la fréquence d'échantillonnage est très élevée.
\end{enumerate}

\subsubsection{Rappel : Méthode de \textit{Tustin}}

\textbf{Approximation par une somme de trapèze :}

$$ u \longrightarrow {\boxed{\int \; = \dfrac{1}{s}}} \longrightarrow y$$

$$ \boxed{ y_k = y_{k-1} + h\cdot \dfrac{u_k + u_{k-1}}{2} } $$

\begin{itemize}
\item On remplace $\boxed{\dfrac{1}{s}}$ par $\dfrac{f(z+1)}{2(z-1)}$\\
\item Remplacer $\boxed{S}$ par $\dfrac{2(z-1)}{h(z+1)}$
\end{itemize}

$\boxed{G_{ca} = tf(\cdots,\cdots)}$ \hfil \textbf{MatLab} : \codeword{c2d(gca,h,'tustin');}

$$ u[k]\longrightarrow\boxed{D/A}\longrightarrow \boxed{G_a(s)} \overbrace{\longrightarrow}^{u(t)} \boxed{A/D} \overbrace{\longrightarrow}^{y(t)} y[k]$$

\figc{1}{182}

$$y[k] \longrightarrow \boxed{H(z)} \longrightarrow y[k] $$

\textbf{Attention : }\\
\begin{itemize}
\item Si \underline{$h$} est \textbf{très petit} $\rightarrow$ privilégié la méthode \textbf{A}
\item Si \underline{$h$} est plutôt \textbf{grand} $\rightarrow$ privilégié la méthode \textbf{B}   
\end{itemize}

\subsubsection{Modèle échantillonné du système à régler}

\figc{1}{183}

\begin{center}
\textbf{Trouver $H(z)$ à partir de $G_a(s)$}
\end{center}

\subsubsection{Calcul du modèle échantillonné dans l'espace d'état}

\figc{1}{184}

$$A_d = e^{A_a\cdot h}$$
$$B_d=\int_{0}^{h}e^{A_a \cdot \tau}B \; d\tau$$

$$
\begin{cases}
\vec{x}[k+1]=A_d\vec{x} [k]+B_d u[k] \\
y[k] = C_d \vec{k} + D_d u[k]
\end{cases} 
$$

\subsection{Conseils pour l'implémentation               (2/2)}
\begin{itemize}
\item \textbf{Simuler d'abord l'effet de quantification }(word length) \codeword{Matlab/Simulink} : "fixed point toolbox" avant d'implémenter.
\item \textbf{Simuler également l'effet de bruit de mesure et l'effet d'incertitude dans les paramètres du système à régler} (erreurs de modélisation) avant d'implémenter.
\item \textbf{La génération de code automatique} peut simplifier l'implémentation (\codeword{Matlab Coder/Embedded Coder/Simulink Coder})
\item \textbf{Si le régulateur comprend un intégrateur pur (}action I), séparez celui-ci, et implémenter de manière séparée avec anti-windup et monitorage (seuil pour déclencher warning etc.)
\item \textbf{Éviter les "formes canoniques commandable/observable"} c-à-d ne pas travailler avec les coefficients des polynômes surtout si l'ordre du régulateur est élevé. Le problème numérique fût observé la première fois par \textit{Wilkinson}.
\item \textbf{Implémentation dans l'espace d'état avec des matrices pleines} est nettement mieux conditionné, mais le temps de calcul est lourd.
\item \textbf{Bon compromis :} implémentation "\textit{BIQUAD}" : mise en cascade de systèmes d'ordre 2, regroupement de pôles et zéros proches dans le même étage \textit{BIQUAD}
\end{itemize}

\subsubsection{Biquad - mise en cascade de fonctions de transfert  d'ordre 2 }

\figc{0.8}{185}

L'avantage de l'implémentation \textit{Biquad} est qu'elle est numériquement plus stable.\\

\textbf{Numériquement stable : }\\
Un système est numériquement stable si la matrice du système comporte le moins possible, et au mieux aucun zéros. Cela évite les erreurs d'arrondis dû aux méthodes de résolutions numériques.\\

\subsection{Motivation : valeurs singulières d'une matrice}

\begin{itemize}
\item Comment déterminer si un système est bien commandable ?\\
(notion quantitative, pas seulement oui/non)
\item Comment déterminer de manière robuste le rang d'une matrice ?\\
(le résultat devrait pas dépendre de petites erreurs de modélisation)
\item Déterminer si l'inversion d'une matrice est bien conditionnée ?
\item Pour un système multivariable (MIMO) qu'est-ce qu'il faudrait tracer dans le diagramme de Bode comme équivalent du module ?
\end{itemize}

$$\boxed{\text{valeurs singulières d'une matrice = amplification directionnelle !}}$$

\subsection{Rappel normes : à quoi ça sert...}

\figc{1}{186}

\subsubsection{Normes de vecteurs}

\figc{1}{189}

\subsubsection{Normes de matrices}



\figc{1}{191}
 
 $\sigma_1$ valeur singulière la plus élevé, demi axe paraboloïde la plus élevé $\rightarrow$ amplification maximal !

\subsection{Décomposition en valeurs singulières (SVD)}

$$ \begin{array}{l l}
\textbf{M} = \textbf{U} \textbf{S} \textbf{V}^T & [\text{U, S, V}]=\text{svd}(\textbf{M}) \\
m \times n \; (m \times m)(m \times n)(n \times n)
\end{array} $$\\

Avec : \textbf{U} et \textbf{V} : matrices orthogonales (unitaires)
$$ 
\begin{array}{c c}
	\begin{array}{l}
		\textbf{U}^T \textbf{U} = \textbf{I} \\
		\textbf{V}^T \textbf{V} = \textbf{I} \\[24pt]
		\textbf{Valeurs singulières :} \\
		\sigma_1 \geq \sigma_2 \geq \cdots \geq \sigma_n \geq 0 \\
	\end{array}
& 
\textbf{S} = 
	\overbrace{\begin{rcases}\begin{bmatrix}
	\sigma_1 & 0 & \cdots & 0 \\ 0 & \sigma_2 & \cdots & 0 \\ \cdots & 0 & \cdots & 0 \\ 0 & 0 & \cdots & 0 \\ 0 & 0 & \cdots & \sigma_n \\ 0 & 0 & 0 & 0 
	\end{bmatrix}\end{rcases}}^{n \text{ col.}} m \text{ lignes}
\end{array} 
$$



\end{document}
