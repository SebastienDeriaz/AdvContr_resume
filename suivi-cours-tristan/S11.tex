\documentclass[document.tex]{subfiles}

\begin{document}

\newpage
\section{S11}

\subsection{Contenue}
\begin{enumerate}
\item Rappel cours 10, exercices série 10
\item Identification fréquentielle
\item Régulation robuste
\end{enumerate}

\subsection{Rappel}
\figc{1}{233}

\subsubsection{Les quatre étapes de l'identification}
\figc{1}{234}
\figc{1}{235}

Il faut impérativement diviser les mesures dans deux groupes séparées :\\

\begin{enumerate}
\item Mesures utilisées pour l'estimation
\item Mesures utilisées pour la validation
\end{enumerate}


\subsection{Least squares (LS) : un problème d'optimisation bien maîtrisé}
\figc{1}{236}



\subsubsection{Interprétation géométrique : moindres carrés}

Paramètres à identifier regroupés dans un vecteur inconnu :\\
\textbf{Paramètres inconnus :} $\vec{\Theta}= \begin{bmatrix} \Theta_1 \\ \vdots \\ \Theta_n \end{bmatrix} $\\

Mesures donnent lieu à une matrice rectangulaire $\Phi$ ayant plus de lignes que de colonnes, et un vecteur y.\\

Système linéaire \textbf{surdéterminé}; on essaie de minimiser le résidu $\Phi \Theta - y$ dans le sens des moindres carrés.

$$
\boxed{\underset{\Theta}{\min} \parallel \Phi \Theta - y \parallel_2}
$$


Solution : le vecteur résidu est perpendiculaire à l'hyperplan engendré
par les colonnes de la matrice $\Phi$.

\figc{1}{238}


\textbf{Vecteur résidu :} $\vec{\varepsilon}$\\

$$
\begin{array}{l l}
\vec{\varepsilon} \perp \vec{\Phi}_1 & \vec{\Phi}_1 \text{: 1\er colonne de }\Phi  \\
\vec{\varepsilon} \perp \vec{\Phi}_2 & \vec{\Phi}_e \text{: 2\eme colonne de }\Phi 
\end{array}
$$

$$
\begin{array}{c}
\vec{\Phi}_1^T \cdot \vec{\varepsilon} = 0 \\
\vec{\Phi}_2^T \cdot \vec{\varepsilon} = 0 \\
\vdots
\end{array}
$$

$$
\boxed {\Phi^T \cdot \vec{\varepsilon} = \vec{0}}
$$

\subsubsection{Solution moindres carrés (least squares)}

\figc{0.4}{239}
\figc{1}{240}

\subsection{Identification dans le domaine fréquentiel}

\textbf{Mesure du diagramme de Bode :}\\
\begin{enumerate}
\item Appliquer fréquence par fréquence, attendre le régime harmonique établi, mesurer le ratio des amplitudes et le déphasage. \\
$\Longrightarrow$ Prend beaucoup de temps pour peu de points fréquentiels
\item Appliquer un signal d'excitation "riche" en fréquences, p.ex. \textbf{SBPA} (suite binaire pseudo-aléatoire), appliquer une \textbf{FFT} sur le signal d'entrée et de sortie, puis diviser.
\item Appliquer un balayage fréquentiel "\textit{sine sweep}"
\end{enumerate}

\subsubsection{Mesure du diagramme de Bode par balayage}

\figc{1}{241}
\figc{1}{242}

\subsubsection{Rappel : régime harmonique numérique}
\figc{1}{243}

\subsubsection{Identification fréquentielle}

On dispose de mesures de la fonction de transfert numérique à certaines 
pulsations, donc on connaît les nombres complexes $$ G_k = G(e^{j \omega_k h}) $$\\

On propose une fonction rationnelle pour  $G(z)$, avec des coefficients inconnus
au numérateur et au dénominateur.\\

On peut facilement formuler un problème aux moindres carrés pour déterminer
les paramètres inconnus.\\

\textit{Voir : Exercice 11.3}


\subsubsection{Exemple exercice 11.3}

$$
G(z) = \dfrac{K}{z(z+a)}=\dfrac{y(z)}{U(z)}
$$

\textbf{Paramètres inconnus :} $\vec{\Theta}= \begin{bmatrix} \Theta_1 \\ \vdots \\ \Theta_n \end{bmatrix} $\\

$$
\begin{cases} 3\Theta_1 + 5\Theta_2 = 1 \\ \Theta_1 + 7\Theta_2 = 3 \end{cases} \rightarrow \text{Nb. éq.} = n
$$

$$
\begin{array}{l l l}
\Phi = \begin{bmatrix} 3 & 5 \\ 1 & 7 \end{bmatrix} \quad y = \begin{bmatrix} 1 \\ 3 \end{bmatrix} & \begin{cases} \Phi\vec{\Theta} = \vec{y} \\ \Theta = \Phi^{-1} y \end{cases} & \Phi \vec{\Theta} - \vec{y} = \underbrace{\vec{0}}_{\text{résidue}}
\end{array}
$$

$$
\underset{\Theta}{\min} \parallel \underbrace{\Phi \vec{\Theta}-\vec{y}}_{\vec{r}} \parallel_2 \quad \vec{r} = \Phi \vec{\Theta}-\vec{y}=\begin{bmatrix}
r_1 \\ r_2 \\ \vdots \\ r_n
\end{bmatrix} \quad \parallel\vec{r}\parallel_2= \sqrt{\vec{r_1^2 + r_2^2 + \ldots r_n^2}}
$$

$ \Theta_{\text{opt}} $ ? :\\

\begin{tabular}{l l l}
$\Phi$ : $n \times p$ & $\Phi^T$ : $p \times n$ & $\Phi^T \Phi$ : $p \times p$ (carrée)
\end{tabular}

$\\Theta = \begin{bmatrix} K \\ a \end{bmatrix}$ :\\

$$  
\begin{array}{c}
	u[k]\underset{\text{connu}}{\longrightarrow}\boxed{G(z)}\underset{\text{mesure bruitée}}{\longrightarrow y[k]} \\[12pt]
	u[k] = \{1,1,1,\ldots\} \\[6pt]
	z^2 y(z) + z a y(z) = K U(z) \quad \vert z^{-2} \\[6pt]
	y(z) + a z^{-1}y(z) = K z^{-2}U(Z) \vert \mathcal{Z}^{-1}\\[6pt]
	\downarrow \\[6pt]
	\boxed{y[k]+ a y[k-1]=Ku[k-2]}
\end{array}
$$

\subsubsection{Minimisation de l'erreur absolue - problème}

\figc{1}{244}

\subsection{Régulateur robuste}

\subsubsection{tentative de définition "régulation robuste"}

\textbf{Ensemble d'outils permettant d'effectuer :}\\

\begin{enumerate}
\item L'\textbf{analyse} des propriétés d'une boucle fermée avec un système à régler comprenant des \textbf{incertitudes}.
\item la \textbf{synthèse} d'un régulateur \textbf{fixe} (non adaptatif) pour une \textbf{famille} de systèmes à régler avec des \textbf{incertitudes} (paramétriques et non-paramétriques) telle qu'un certain niveau de performance en boucle fermée soit \textbf{préservé} pour toute la famille des systèmes à régler.
\end{enumerate}

\subsubsection{Rappel : Nombre d'encerclements nécessaires pour la stabilité en boucle fermée}

\textbf{Critère de Nyquist généralisé : }\\

$\Rightarrow$ valable pour tous les systèmes stables ou instables en b.o.\\
La \textbf{boucle fermée} est \textbf{stable} si et seulement si le \textbf{nombre
d'encerclements du point critique} $-1$ dans le sens trigonométrique correspond au \textbf{nombre de pôles instables en boucle ouverte}.

\figc{1}{245}

\subsubsection{Définition de la distance critique}

\figc{1}{246}

\subsubsection{Marges de stabilité}

Une \textbf{distance critique} confortable (p.ex $d_{\text{crit}} > 0.5$) implique des 
bonnes marges de gain et de phase.

$$
\boxed{ A_m > \dfrac{1}{1-d{\text{crit}}}} \quad \quad \quad \boxed{\phi_m > 2 \arcsin\bigg( \dfrac{d_{\text{crit}}}{2} \bigg)}
$$

\textbf{Attention : Le contraire n'est pas vrai !}\\

\figc{0.5}{247}

\subsubsection{Rappel normes : à quoi ça sert}
\figc{1}{248}

\subsubsection{Normes pour matrices}
\figc{1}{249}

\subsubsection{Vecteurs - Matrices - Signaux - Systèmes LTI}
\figc{1}{250}

\subsubsection{Norme $H_2$ , lien temporel/fréquentiel : thrm. de Parseval}
\figc{1}{251}

L'énergie d'un signal dans le domaine temporelle est égale à l'énergie dans le domaine fréquentiel, soit plus précisément :

\begin{center}
\textit{La norme $H_2$ de $G(s)$ correspond à la norme $L_2$ de sa réponse impulsionnelle $g(t) =$ énergie de la réponse impulsionnelle}
\end{center}

\subsubsection{norme $H_\infty$ = norme induite $L_2$ : \textit{worst case}}
\figc{1}{252}










\end{document}

