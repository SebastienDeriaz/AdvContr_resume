\documentclass[document.tex]{subfiles}

\begin{document}

\section{S6 - Programme Advanced Control}

\subsection{Contenue}

\figc{1}{144}

\subsection{Rappel}

\subsubsection{Critère de Nyquist simplifié : marge de gain et de phase}

\textbf{Par diagramme de Bode : }
$$\phi_m = \text{arg}(G_o(j\omega_{co}))+ \pi \; \longrightarrow \; |G_o(j\omega_{co})|=1 $$

\figc{1}{145}

\subsubsection{Critère de Nyquist simplifié}

\figc{1}{146}
\figc{1}{147}

\subsubsection{Boucle ouverte pour le retour d'état}

\figc{1}{148}

\subsubsection{Robustesse du régulateur LQR (retour d'état)}

\figc{1}{150}
\figc{1}{149}

$$d_{\text{crit}}= \underbrace{\text{min}}_{\omega} \; \text{dist}(\underbrace{-1}_{\text{point crit.}},L(j \omega)) $$

Afin d'assurer une bonne \textbf{robustesse} : \\
$\Rightarrow$ Avec synthèse \textbf{LQR} : $\longrightarrow d_{\text{crit}} = 1$

\subsubsection{Régulation en correspondance LQR avec action intégrale}
\figc{1}{151}

\subsection{Observateur}

\subsubsection{Motivation}

\begin{enumerate}
\item Le retour d'état nécessite la mesure de \textbf{toutes} les variables d'état
\begin{enumerate}
\item Cher et compliqué, car beaucoup de capteurs nécessaires
\item Peut poser des problèmes de fiabilité dû au nombre de capteurs
\item Parfois pas faisable s'il y a des variables d'état qui sont difficiles à mesurer
\end{enumerate}
\begin{itemize}
\item Idée 1 : estimer les variables d'état manquantes, puis utiliser ces signaux estimés pour le retour d'état
\item idée 2 : cet observateur/estimateur nécessite la connaissance d'un modèle du processus physique et intègre ce modèle
\end{itemize}
\item Il y a des applications où on dispose de plusieurs mesures bruitées, p.ex. IMU (Inertial Measurement Unit) avec accéléromètres 3 axes, gyroscope 3 axes, magnétomètres 3 axes, signaux GPS, etc.
\begin{itemize}
\item Idée 3 : un observateur permet de "consolider" toutes ces informations imparfaites, et de fournir une information consolidée plus précise et moins bruitée "sensor fusion".
\end{itemize}
\end{enumerate}

$$\boxed{\begin{array}{c}
\text{Il y a donc une vision \textbf{déterministe} d'un observateur, et une vision stochastique.}\\ \text{Dans le contexte \textbf{stochastique}, l'observateur est appelé aussi "\textbf{filtre de Kalman}"}
\end{array}}$$

\subsubsection{Exemple : Observateur pour monitorer des signaux difficilement mesurables}

\figc{1}{153}

\subsection{Observateur "trivial"}
\subsubsection{Idée naïve : observateur "trivial"}

\figc{1}{154}

\subsubsection{Dynamique de l'erreur de l'observateur trivial}

\figc{1}{155}

\begin{itemize}
\item La dynamique de l'erreur est indépendante du signal de commande u(t).
\item $x_e(t) = e^{At}x_{e,0}$ : sous quelle condition l'observateur trivial converge ?
\item Converge que si le processus est stable, i.e. si toutes les valeurs propres
\end{itemize}

\begin{center}
\textit{\textbf{Il faut que les valeurs propres de la matrice $A$ soit négatif \\ $\longrightarrow$ Système stable !}}
\end{center}

\subsubsection{Points faibles de l'observateur trivial}

\begin{itemize}
\item Le signal de sortie mesuré $y(t)$ du processus n'est pas utilisé !
\item On est complètement tributaire de la dynamique du processus, aucun moyen d'influencer le régime transitoire de l'observateur.
\end{itemize}

\textit{\textbf{$\Rightarrow$ Pour ces raisons l'observateur trivial n'est jamais utilisé !}}\\

\textbf{Solution : }
\begin{enumerate}
\item Comparer ce que \textbf{l'on a} avec ce que \textbf{l'on veut} !
\item Utiliser un \textbf{feedback} de l'erreur, c-à-d un \textbf{rebouclement} pour corriger le tir !
\end{enumerate}

\subsection{Observateur complet}
\subsubsection{Structure de l'observateur complet}

\figc{1}{156}

\subsubsection{Dynamique de l'erreur de l'observateur complet}

\figc{1}{157}

\textbf{Comparaison retour d'état et observateur :}


\begin{center}
\begin{tabular}{|| l c l ||}
\hline \hline
\textbf{Retour d'état}	& : 	& $A_{\text{bf}}=A-B\overbrace{\boxed{K}}^{\text{vec. lig.}}$ \\[6pt]
\textbf{Observateur}		& :	& $A_{\text{obs}}=A-\underbrace{\boxed{H}}_{\text{vec. col.}}C$ \\[18pt] \hline \hline
\end{tabular}
\end{center}


$$\boxed{
\begin{array}{c c c}
\dot{x}_e(A-HC)x_e & \text{avec} & x_e = x - \hat{x}
\end{array}
}$$

\begin{itemize}
\item Pour le choix H = 0 on retombe sur l'observateur trivial !
\item L'observateur converge si la matrice \underline{$A-HC$} est stable, c-à-d si les valeurs propres de \underline{$A_{\text{obs}} = A − HC$} ont toutes une partie réelle négative.
\item Le gain \underline{$H$} de l'observateur est à déterminer : "\textbf{synthèse de l'observateur}". Le choix de \underline{$H$} détermine la dynamique de l'observateur.
\item Le \textbf{placement de pôles} est \textbf{une} possibilité pour la synthèse de l'observateur.
\end{itemize}

\subsection{Dualité : synthèse retour d'état / synthèse observateur}

\figc{1}{158}

\begin{flushleft}
\textbf{Interprétation stochastique} : \textit{\textbf{LQG} = \textbf{L}inear \textbf{Q}uadratic \textbf{G}aussian}
\begin{itemize}
\item \textbf{Q} : covariance du bruit de processus
\item \textbf{R} : covariance du bruit de mesure gain optimal
\item \textbf{H} de l'observateur pour minimiser la variance de l'erreur
\end{itemize}
\end{flushleft}

\subsection{Condition observabilité}

\subsubsection{Retrouver la condition initiale}

\figc{1}{159}

\subsubsection{Condition d'observabilité}

$$
\boxed{\text{rank}[P_o]=\text{rank} \begin{bmatrix} C \\ CA \\ \vdots \\ CA^{n-1} \end{bmatrix} = n}
$$

$$
\boxed{\begin{array}{c}
\text{Un système est \textbf{observable} si le \textbf{rang} de la \textbf{matrice d'observabilité est \textbf{\textit{n}}}} \\[6pt]
 \Rightarrow \boxed{\text{rank}(P_o) = n} \\[6pt]
\text{En \textbf{monovariable} (SISO) : si } \text{det}(P_o)\neq 0
\end{array}} 
$$

\subsection{Décomposition de Kalman}

\subsubsection{Réalisation minimale}

Chaque \textbf{système LTI} peut se décomposer après un changement de variables en :
\begin{itemize}
\item Une partie qui est \textbf{commandable} et \textbf{observable}
\item Une partie qui n'est \textbf{pas commandable}, mais \textbf{observable}
\item Une partie qui est \textbf{pas observable} mais \textbf{commandable}
\item Une partie qui est \textbf{ni commandable}, \textbf{ni observable}
\end{itemize}

Les parties non commandable et/ou non observable ne contribuent rien au comportement entrée-sortie; les variables d'état associées peuvent être écartées.\\

On appelle une « \textbf{réalisation minimale} » d'une matrice de transfert une représentation dans l'espace d'état avec un nombre de variables d'état minimal.\\

\textit{\textbf{\underline{Une telle réalisation minimale est commandable et observable !}}}

\subsubsection{décomposition de Kalman}

\figc{1}{160}

\subsection{Exemple - observateur comme filtre complémentaire}

\textbf{Navigation inertielle en utilisant un accéléromètre (inclinomètre) et un gyromètre}

\figc{0.5}{161}

\begin{enumerate}
\item \textbf{L'accéléromètre} basé sur une masse sismique qui mesure la gravité (composante DC) et les composantes dynamiques de l'accélération.\\
$\longrightarrow$ Le signal fourni y(t) n'est que fiable à \textbf{basse fréquence}.
\item \textbf{Le gyromètre} mesure une vitesse de rotation u(t).\\
Pour connaître l'angle il faudrait intégrer u(t), mais ceci risque de créer une dérive (instable).\\
$\longrightarrow$ Le signal fourni u(t) n'est que fiable à \textbf{haute fréquence}.
\end{enumerate}

\subsubsection{Application : filtre complémentaire}

$
\begin{array}{l c l}
U(s) & : & \text{gyromètre (vitesse)} \\
Y(s) & : & \text{accéléromètre (inclinomètre)}\\
\hat{Y}(s) & : & \text{information consolidée} \\
\alpha & : & \text{angle d'inclinaison}\\[12pt]
\end{array}
$

$$\boxed{
\underbrace{F_1(s)}_{\text{HP filter}} \cdot \underbrace{\frac{1}{S}}_{\text{intégrateur}} \cdot \; U(s) + \underbrace{F_2(s)}_{\text{LP filter}} \cdot \; y(s) = \underbrace{\hat{y}(s)}_{\text{est. dyn. de }\alpha}
}$$

$$ \Rightarrow \boxed{F_1(s)} + \boxed{F_2(s)} = 1 \; \text{(filtre complémentaire)} $$
\figc{1}{162}

\subsubsection{Observateur pour processus « intégrateur pur »}

\figc{1}{163}

\subsubsection{Fonctions de transfert du filtre}

\figc{1}{164}

\subsection{Interprétation stochastique de l'observateur}

\begin{enumerate}
\item S'il y a \textbf{peu de bruit} sur la mesure $y(t$), et que je peux faire confiance, je peux choisir un \textbf{gain élevé} \underline{$H$} pour l'observateur.\\
Ce \textbf{gain élevé} \underline{$H$} donne une \textbf{pulsation de coupure élevée} du filtre.
\item D'autre part, si la mesure de $y(t)$ est \textbf{très bruitée}, et que je peux faire plus confiance au modèle, je peux choisir une valeur \textbf{faible pour le gain} \underline{$H$}.\\
Dans ce cas, le filtre aura une \textbf{pulsation de coupure basse}.
\end{enumerate}

\figc{1}{165}

\subsection{Principe de séparation}

\subsubsection{Rappel}

$$ \begin{array}{c c}
\boxed{\dfrac{\frac{1}{s}}{1+\frac{H}{s}} = \dfrac{1}{s+H}} & \boxed{\dfrac{\frac{H}{s}}{1+\frac{H}{s}} = \dfrac{H}{s+H}} \\[18pt]
\begin{cases} \dot{x} = Ax + Bu \\ y = Cx\end{cases} & \boxed{G(s) = C(sI-A)^{-1}B}
\end{array} $$

\subsubsection{Principe de séparation}

$$
\begin{bmatrix} \dot{x} \\ \dot{\tilde{x}} \end{bmatrix} = \begin{bmatrix} A - BK & BK \\ 0 & A-HC \end{bmatrix} \begin{bmatrix} x \\ \tilde{x} \end{bmatrix}
$$

\textbf{La matrice A en boucle fermée est bloc triangulaire :}
$$
\begin{array}{c}
\text{det}\begin{bmatrix} sI - A + BK & - B K \\ 0 & sI - A + HC \end{bmatrix} = 0 \\[12pt]
\underbrace{\text{det}(sI - A + BK)}_{\text{pôles du retour d'état}} \cdot \underbrace{\text{det}(sI - A + HC)}_{\text{pôles de l'observateur}}
\end{array}
$$

\begin{flushleft}
\begin{tabular}{|c|}
\hline
\textit{\textbf{Les pôles de la boucle fermée sont composés des pôles de la synthèse}}\\
\textit{\textbf{du retour d'état et des pôles de la synthèse de l'observateur !}} \\ \hline
\end{tabular}
\end{flushleft}

\subsubsection{Explication}
\begin{conditions}
	Processus & $\dot{x} = A x + B u$ \\
	Feedback & $u = -K\hat{x}$\\
	Observateur & $\dot{\hat{x}}= A \hat{x} + B u + HC(x-\hat{x})$
\end{conditions}

$$
 \begin{array}{l r}
 x_{tot} = \left. \begin{bmatrix} x \\ \hat{x}\end{bmatrix} \right \} 2\cdot n & \begin{cases} \dot{x}=Ax - BK\hat{x} \\ \dot{\hat{x}} = A\hat{x + Bu + HC(x-\hat{x})} \end{cases}
 \end{array}
$$

$$
\begin{array}{l c r}
\dot{x}_{tot} = A_{tot} & \text{avec} & A_{tot} = \begin{bmatrix} A & -BK \\ HC & A-BK-HC \end{bmatrix}
\end{array} 
$$ 

on peut ensuite calculer les valeurs propre de $A_{tot}$ $\longrightarrow$ pôles du systèmes \\

\textbf{Alternative : } 
$$ \begin{array}{l r}
 x_e = x-\hat{x} \Rightarrow \hat{x} = x - x_e 
& \rightarrow \text{remplacer le } \hat{x} \text{ par } x_e \\[12pt]
 \Rightarrow x_{tot} = \begin{bmatrix} x  \\ x_e\end{bmatrix} & \Rightarrow A_{tot} = \underbrace{\begin{bmatrix}
A-BK & BK \\ 0 & A-HC\end{bmatrix}}_{\text{Matrice Triangulaire}}
\end{array} $$

Les v\textbf{aleurs propres} de \underline{$A_{tot}$} sont les \textbf{valeurs propres} de \underline{$A-BK$} et les \textbf{valeurs propres} de \underline{$A-HC$}


\subsubsection{Remarques importantes}

\begin{itemize}
\item Le principe de séparation n'est que valable si le modèle est parfaitement connu.
\item  réglage par \textbf{retour d'état LQR} est \textbf{robuste} si toutes les variables d'état sont mesurées (pas besoin d'observateur).\\
Marges de stabilité : $\boxed{\phi_m > 60 \degres, A_m = [0.5 \ldots  \infty]}$
\item En utilisant un observateur, \textbf{cette propriété de robustesse est perdue}. La méthode \textbf{LTR} essaye de récupérer cette propriété, mais au dépens d'un gain $H$ de l'observateur important (amplification du bruit).
\item  Il y a des systèmes à régler \textbf{intrinsèquement} difficile à régler, où \textbf{aucune} méthode de synthèse fournit une bonne robustesse !\\
Exemple : systèmes avec déphasage nonminimal (zéros positifs)
\end{itemize}

\subsection{« Model order reduction »}

\figc{1}{175}

\subsection{LTR : Loop Transfer Recovery}

\figc{1}{173}

LTR (\textit{Loop Transfer Recovery}) est une méthode de choix des pondérations $Q$ et $R$ de l'observateur, permettant de récupérer la bonne robustesse du retour d'état :
$$ \boxed{\phi_m > 60 \degres,\; A_m = [0.5 \; \ldots \; \infty]} $$

\begin{center}
\begin{tabular}{l l}
$Q = \rho B B^T$ et $R=1$ & $\rho \longrightarrow \infty$ : Observateur rapide
\end{tabular}
\end{center}


\textbf{\underline{Attention :} Ne fonctionne que pour les système à déphasage minimale !} \\

\textbf{Système à déphasage minimal :}\\
$\Leftrightarrow$ \textbf{Pas de zéro} dans le \textbf{demi-plan droite}.

\textbf{Boucle ouverte :}
$$ \underline{\hat{G}_0(s)=C(sI-A)^{-1}B\cdot K(sI-A+BK + HC)^{-1}H} $$


\subsubsection{lieu de Nyquist en fonction de $\rho$}

\figc{1}{176}

\subsection{Résumé de la synthèse LQR - LQG - LTR }

\begin{itemize}
 \item Basé sur un modèle dans l'espace d'état \textbf{suffisamment précis} \\ (nécessite une bonne modélisation plus identification des paramètres)
 \item Condition nécessaire : le système doit être \textbf{suffisamment bien commandable et observable}
 \item Basé sur un critère d'optimisation \textbf{quadratique} \\ (il existe bien d'autres critères)
 \item On "joue" avec les pondérations $Q$ et $R$ du critère, pas directement avec les paramètres du régulateur. \\ En règle générale, plusieurs itérations sont nécessaires.
 \item Méthode assez systématique et bien documentable (contrairement au loop shaping manuel (p.ex. utilitaire Matlab sisotool), où on bidouille les paramètres du régulateur).
 \item La méthode garantit que la \textbf{boucle fermée nominale} est stable ! 
 \item La méthode fonctionne également pour les systèmes multivariables(MIMO)
\end{itemize}

\textbf{La méthode LQR / LQG / LTR n'apporte rien pour : }
\begin{itemize}
\item Des systèmes monovariables de faible ordre (p.ex. ordre 2), car on peut aussi bien faire avec un PID classique 
\item Des systèmes mal connus ayant beaucoup d'incertitudes de modélisation
\end{itemize}


















\end{document}
