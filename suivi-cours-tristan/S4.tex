\documentclass[document.tex]{subfiles}

\begin{document}

\section{S4}

\subsection{Contenue cours S4}

\begin{enumerate}
\item Rappel cours 3, exercices série 3
\item Réalisation minimale d'un système
\item Application de méthodes d'optimisation dans la régulation 
\item Formes quadratiques
\item Retour d'état par la méthode LQR
\item Équation de Riccati, cas temps continu
\item Équation de Riccati, cas temps discret
\item Exemple ordre 1
\end{enumerate}

\subsection{Rappels cours 3}

\begin{itemize}
\item Retour d'état
\item Placement de pôles pour le dimensionnement du retour d'état
\item Commandabilité
\item Forme commandable
\end{itemize}

\subsection{Retour d'état  « \textit{state feedback} »}

\textbf{Rendre la boucle fermée \underline{stable} avec une « \textit{dynamique} » appropriée}

\figc{1}{74}

\subsection{Commandabilité}

\figc{1}{75}

La commandabilté  signifie que n'importe quel point de l'espace d'états est atteignable.

\subsubsection{Critère algébrique pour la commandabilité}

\figc{0.9}{76}

Un système $[A,B]$ est commandable si sa matrice de commandabilité associée
$P_c$ est de rang $n$.\\
Si la matrice est rectangulaire, il faut qu'elle soit de rang $n$, ce qui signifie qu'il y a au moins $n$ colonnes linéairement indépendante. Pour rappel, $n$ est le nombre de ligne du vecteur de commande $\vec{u}$.

\subsubsection{Forme commandable}

\figc{1}{77}

\subsection{Réalisation minimale}

Un système $[A, B, C, D]$ ayant $n$ variables d'état est appelé une « \textit{réalisation minimale }», si on n'arrive pas à trouver une autre réalisation avec moins de $n$ variables d'état, donnant lieu au même comportement entrée-sortie. \\

Cette propriété importante est exigée pour l’application de toutes les méthodes de synthèse basée sur un modèle. \\

Un théorème important dit : Une réalisation $[A, B, C, D]$ est \underline{minimale} si $[A, B]$ est \underline{commandable} et $[A, C]$ \underline{observable} (cette dernière propriété sera traîtée prochainement). \\

Pour un système monovariable (SISO), on peut facilement depuis une fonction de transfert ayant aucun \underline{pôle et zéro en commun} trouver une réalisation minimale (p.ex. forme commandable).

Pour les systèmes multivariables (MIMO), c'est beaucoup plus difficile, la théorie complète existe, mais n'est pas traitée ici.

\subsubsection{Exercice 4.2}

\figc{0.8}{78}

\subsection{Motivation : formes quadratiques}


La régulation avancée est basée sur :
\begin{itemize}
\item L'utilisation explicite d'un \textbf{modèle} dans la synthèse du régulateur
\item L'utilisation d'un critère \textbf{d'optimalité} (on cherche le meilleur compromis). On veut minimiser une fonction coût.\\
\end{itemize}

\textbf{Les fonctions coût qu'on vise à minimiser sont souvent des « \textit{formes quadratiques} »}\\

\textbf{Pourquoi ?}\\
Trouver le minimum d'une fonction revient à annuler sa dérivée, et le gradient d'une forme quadratique est une fonction linéaire $\rightarrow$ solution analytique

\subsection{Rappel des formes quadratiques dans $\mathbb{R}^n$}

Une forme quadratique est un polynôme homogène de degré 2 à plusieurs variables. Exemples :

\figc{0.6}{79}

La norme euclidienne au carré est un autre exemple d'une forme quadratique :

\begin{equation}
\vline \; \vline (x,y,z) \vline \; \vline^{\;2} = x^2 + y^2 + z^2
\end{equation}

\textbf{Une forme quadratique est définie avec l'aide d'une matrice M \underline{symétrique}}

\begin{equation}
\begin{array}{c | c}
\text{\textbf{Symetrique}} & \text{\textbf{Antisymétrique}} \\[2pt] \hline \; \\[-8pt] 
M = M^T & M = -M^T \\
\; & M = \begin{bmatrix} 0 & M_{12} \\ -M_{12} & 0 \end{bmatrix} \\[12pt]
\; &  x^TMx = \textcolor{red}{0}
\end{array}
\end{equation}

\begin{equation}
\begin{array}{c}
M = \begin{bmatrix} M_{11} & M_{12} \\ M_{21} & M_{22} \end{bmatrix}, \; x = \begin{bmatrix}
x_1 \\ x_2 \end{bmatrix} \\[24pt]
x^T M x = \begin{bmatrix} x_1 x_2 \end{bmatrix} \cdot \begin{bmatrix} M_{11} x_1 & M_{12} x2 \\ M_{21} x_1 & M_{22} x_2 \end{bmatrix} \\[24pt]
x^T M x =\underbrace{M_{11} x_1^2 + 2 M_{12}x_1 x_2 + M_{22} x_2^2}_{\text{Polynome homogène de degrée 2}}
\end{array}
\end{equation}

\textbf{Cas général :}

\begin{equation}
Q(x_1,\ldots,x_n)=\sum_{i=1}^n \sum_{i=1}^n m_{i j}x_1 x_j = x^t M x
\end{equation}

Une matrice antisymétrique avec diagonale nulle ne génère aucune contribution à la forme quadratique.


\subsection{Formes quadratiques définies}

\figc{0.7}{80}

On aimerait savoir si le résultat est positif pour tout vecteur x.

\figc{0.9}{81}

\subsection{Matrices définies positives}

On écrit  $M > 0$  si  $x^t M x>0$, $ \forall x \neq 0$ \\

La forme quadratique donne une surface courbée vers le haut partout :

\figc{0.4}{82}

\subsection{Critère de Sylvester pour déterminer si $M = M^t > 0$}

Une matrice symétrique est définie positive, si tous les mineurs diagonaux principaux sont positifs, en partant du coin supérieur gauche. Exemple :

\figc{0.6}{83}

\subsection{Propriété d'une matrice symétrique définie positive}

Toutes les valeurs propres d'une matrice symétrique définie positive sont réels et positifs.\\

Cette propriété est une condition nécessaire et suffisante.

\subsection{Matrices semi-définies positives}

On écrit $ M \geq 0$  si  $x^t M x \geq 0$,  $\forall x \neq 0$ \\

La forme quadratique donne une surface courbée vers le haut presque partout, mais il y a des directions dans lesquelles la courbure est nulle !

\figc{0.5}{84}

\subsection{Catégories de matrices}

\begin{itemize}
\item Semi-définie positive : Forme quadratique $x^t M x \geq 0$, courbure positive
\item Définie positive : Forme quadratique positive pour chaque x, courbure positive
\item Semi-définie négative : Forme quadratique $x^t M x \leq 0$, courbure négative
\item Définie négative : Courbure négative partout
\item Indéfinie (point de selle): Forme de selle, courbe positive et négative
\end{itemize}

\subsection{Motivation LQR (\textit{Linear Quadratic Regulator})}

\figc{0.8}{85}

La quantité $\boxed{\int_0^\infty x^2(t)dt}$ traduit les défauts du régime transitoire.


\subsection{Qualité du régime transitoire en boucle fermée}


\figc{0.9}{86}

\subsection{Le prix à payer  ...}

\figc{0.9}{87}

\subsection{LQR (\textit{Linear Quadratic Regulator})}


\figc{0.9}{88}

\subsection{Solution LQR analogique - équation de Riccati}

\textbf{Rappel :}
\begin{itemize}
\item \underline{Q} : Pondère le régime transitoire (\textbf{rapidité}) 
\item \underline{R} : Pondère l'énergie nécessaire du signal de contrôle $u$ (\textbf{Consommation})
\end{itemize}
\begin{center}
\tbox{\textbf{Il faut donc faire un compromis entre $Q$ et $R$.}}
\end{center}

\figc{0.9}{89}

\subsection{Propriétés de la régulation LQR}

\figc{0.9}{90}

\subsection{Un choix pour la matrice de pondération Q}

Au lieu de pénaliser le régime transitoire du vecteur d'état x(t), on peut aussi pénaliser un signal de sortie mesuré y(t).


\figc{0.4}{91}

\subsection{\textit{Cheap control - expensive control}}

\begin{itemize}[$\bullet$]
\item Le cas extrême : $R = 0$ est appelé « \textbf{\underline{cheap control}} » \\
$\rightarrow$ Le gain du retour d'état va tendre vers infini !

\item L'autre cas extrême : Q = 0 est appelé « \textbf{\underline{expensive control}} » \\
$\rightarrow$ Il montre une propriété remarquable, voir exercice 10 du polycopié anglais « state regulator », énoncé page 54.
\end{itemize}


\subsection{DLQR (\textit{Discrete Linear Quadratic Regulator})}

\figc{0.9}{92}

\textbf{Fonction coût à minimiser (indice de performance) : }

\figc{0.9}{93}

\subsection{Solution LQR discret - équation de Riccati}

\figc{0.9}{94}

\subsection{Remarques}

\figc{1}{95}

\subsubsection{Exercice 10, p. 54 polycopié anglais}


\figc{1}{96}
\figc{1}{97}
\figc{1}{98}
\figc{1}{99}


\end{document}
