\documentclass[resume]{subfiles}


\begin{document}
\section{Identification}
\subsection{Méthode des moindres carrés}
\begin{enumerate}
\item Poser une représentation du système en $z$
\item Décomposer en équation aux différences
\item Poser l'équation à résoudre avec les moindres carrés
\item Effectuer la pseudo-inverse pour résoudre
\end{enumerate}
Par exemple, la fonction de transfert
$$G(z)=\frac{b}{z-a}$$
Se traduit en
$$y[k]=ay[k-1]+bu[k-1]$$
Et donc
$$\Theta=\begin{bmatrix}
a\\b
\end{bmatrix}$$
$$\Phi=\begin{bmatrix}\\
y[k-1] & u[k-1]\\
\\
\end{bmatrix}$$
Et on résout
$$\Phi\Theta=y\Longrightarrow \boxed{\Theta=\Phi^{+}y}$$
$$\boxed{\Phi^+=\left(\Phi^T\Phi\right)^{-1}\Phi^T}$$
\subsubsection{Identification dans le domaine de $s$ ou $z$}
Il est aussi possible de faire une identification en posant une équation de la forme
$$Ga_1z+Ga_2-b_0z-b_1=-Gz^2$$
Ce qui permet de construire la matrice $\Theta$ des inconnues, et la matrice $\Phi$ des valeurs connues.
$$\Phi=\begin{bmatrix}\\
Gz& +G& -z& -1\\
\\\end{bmatrix}$$

$$\Theta = \begin{bmatrix}a_1\\a_2\\b_0\\b_1\end{bmatrix}$$
Attention ! Pour avoir un résultat réel à la fin pour $\Theta$, il faut utiliser le complexe conjugué

$$\Phi'=\begin{bmatrix}\Phi\\\Phi^\ast\end{bmatrix}\qquad y'=\begin{bmatrix}y\\y^\ast\end{bmatrix}$$
\subsection{Filtres}
\begin{itemize}
\item Passe-bas
$$H(j\omega)=\frac{1}{1+\frac{1}{Q}j\frac{\omega}{\omega_0}+\left(j\frac{\omega}{\omega_0}\right)^2}$$
$$G(z)=\frac{b_0z+b_1}{z^2+a_1z+a_2}$$
\item Passe-haut
$$H(j\omega)=\frac{\left(j\frac{\omega}{\omega_0}\right)^2}{1+\frac{1}{Q}j\frac{\omega}{\omega_0}+\left(j\frac{\omega}{\omega_0}\right)^2}$$
\item Passe-bande
$$H(j\omega)=\frac{\frac{1}{Q}j\frac{\omega}{\omega_0}}{1+\frac{1}{Q}j\frac{\omega}{\omega_0}+\left(j\frac{\omega}{\omega_0}\right)^2}$$
\item Coupe-bande
$$H(j\omega)=\frac{1-\left(\frac{\omega}{\omega_0}\right)^2}{1+\frac{1}{Q}j\frac{\omega}{\omega_0}+\left(j\frac{\omega}{\omega_0}\right)^2}$$
\end{itemize}















\end{document}