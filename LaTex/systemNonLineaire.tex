\documentclass[resume]{subfiles}
\begin{document}
\section{Système non linéaire}

\subsection{sans action de l'entrée}
$$\dot{\vec{x}} = \vec{f}(\vec{x})$$ linéarisé autour d'un point de fonctionnement.

$\vec{f}(\vec{x_e})= \vec{0}$ avec $\Delta\vec{x} = \vec{x}-\vec{x_e}$ alors on a $\dot{\Delta\vec{x}} = A \cdot\Delta\vec{x}$ ou $A$ est la matrice jacobienne

\paragraph{}
Stabilité locale: on prend le système linéarisé  $\dot{\Delta\vec{x}} = A \cdot\Delta\vec{x}$ et on calcul les pôles du système si la partie réel des valeurs propre est négative. $\lambda_k = det(\lambda I-A)$ 

\begin{itemize}
\item système linéarisé stable => point d'équilibre du système non-linéaire \textbf{localement stable}.
\item système linéarisé instable => point d'équilibre du système non-linéaire est localement instable.
\item système linéaire marginalement stable => aucune information sur la stabilité du système non-linéaire.
\end{itemize}

\subsection{avec l'action de l'entrée}
$$\dot{\vec{x}} = \vec{f}(\vec{x}, u),\space \vec{f}(\vec{x_e},u_e)=\vec{0}$$
$$A = \frac{\partial\vec{f}}{\partial\vec{x}}\big|_{\vec{x_e},u_e}$$
$$B = \frac{\partial\vec{f}}{\partial u}\big|_{\vec{x_e},u_e}$$

$$\dot{\Delta\vec{x}} = A \cdot\Delta\vec{x}+ B\cdot\Delta u$$
Si plusieurs entrées alors $u$ peut être un vecteur aussi.




\end{document}