\documentclass[resume]{subfiles}
\begin{document}
\section{Nyquist}

\subsection{Nyquist simplifié}
Il faut que la boucle ouverte soit stable sinon il faut le critère généralisé
$$G_o(s) = G_a(s)\cdot G_c(s)$$
$$G_{yw}(s)=\frac{G_o(s)}{1+G_o(s)}$$

\subsubsection{boucle ouverte}
$$G_o(s) = K(sI - A)^{-1}B$$

\subsubsection{marge de phase}
\begin{enumerate}
\item identifier module = 0
\item à la même pulsation regarder la diff entre -180° et la phase actuelle
\end{enumerate}

\subsubsection{marge de gain}
\begin{enumerate}
\item identifier la phase à -180°
\item à la même pulsation regarder la diff entre le module et 0[dB]
\end{enumerate}

\subsection{Nyquist généralisé}
Utilisable en tout temps. La boucle ouverte doit exactement encercler le point critique dans le sens trigonométrique le nombre de pôle instable.
Pour compter le nombre d'encerclement, il faut fixer un élastique sur le point critique et l'autre bout suit la courbe de $-\infty$ à $\infty$. 

\subsubsection{marge de phase et de gain}
Tracer un cercle unité centré à l'origine et check les intersections entre lieux de nyquist et le cercle le plus petit angle nous donne la marge de phase.


\subsection{Distance critique}
Distance entre le point critique (-1) et la courbe du lieux de Nyquist.
$$d_{crit} = min(dist(-1,L(j\omega)))$$




\end{document}