\documentclass[resume]{subfiles}



\begin{document}
\section{Espaces d'états}
\begin{figure}[H]
\centering
\includegraphics[scale=1,page=1]{drwg_0.pdf}
\end{figure}
\subsection{Choix des variables d'état}
Les variables d'état sont les variables qui ont leurs dérivée dans les équations.
\begin{itemize}
\item Condensateur : tension
\item Bobine : courant
\end{itemize}

\subsection{Forme modale}
Matrice $T$ construite à partir des vecteurs propres de $A$
$$T=\begin{bmatrix}
\\
\vec{v}_1 & \vec{v}_2 & \cdots\\
\\
\end{bmatrix}$$
$$\boxed{\begin{split}
\tilde{A} &= T^{-1}AT &\qquad \tilde{B}&=T^{-1}B\\
\tilde{C} &= CT&\qquad \tilde{D}&=D\end{split}}$$
\subsection{$A,B,C,D \longrightarrow G$}
$$G(s)=C(sI-A)^{-1}B+D$$
$$G(z)=C_n(zI-A_n)^{-1}B_n+D_n$$
\subsubsection{Gain haute fréquence}
$$\lim_{s\to\infty}G(s)=D$$
\subsubsection{Gain basse fréquence (gain statique)}
Analogique : $$G(s=0)=-CA^{-1}B+D$$
Numérique  : $$G(z=1)=-C(I-A)^{-1}B+D$$

\subsection{$G\longrightarrow A,B,C,D$}
On utilise la forme commandable
$$G(s)=\frac{b_2s^2+b_1s+b_0}{s^3+a_2s^2+a_1s+a_0}$$
$$\boxed{\begin{split}
A&=\begin{bmatrix}0 & 1 & 0\\0 & 0 &1\\-a_0 & -a_1 & -a_2\end{bmatrix} &
B&= \begin{bmatrix}
0\\0\\1
\end{bmatrix}\\
C&=\begin{bmatrix}b_0 & b_1 & b_2\end{bmatrix} & D&=0\end{split}}$$
\subsection{$G(s) / G(z) \longleftrightarrow A,B,C,D$}
\begin{figure}[H]
\centering
\includegraphics[scale=1,page=1]{drwg_1.pdf}\\
\includegraphics[scale=1,page=2]{drwg_1.pdf}
\end{figure}

\subsection{Conversion analogique vers numérique}
$$A_d = e^{A_a\cdot h}$$
$$B_d = \int_0^h e^{A_a\cdot\tau}B_a d\tau$$
$$C_d = C_a$$
$$D_d = D_a$$

\subsection{Mise en cascade}
\begin{figure}[H]
\centering
\includegraphics[scale=1,page=2]{drwg_0.pdf}
\end{figure}
$$S_{tot}=S_2(s)\cdot S_1(s)\qquad\text{ordre important}$$
$$A_{tot}=\begin{bmatrix}
A_1 & 0\\
B_2C_1 & A_2
\end{bmatrix}\qquad B_{tot}=\begin{bmatrix}
B_1\\B_2D_1
\end{bmatrix}$$
$$C_{tot}=\begin{bmatrix}
D_2C_1 & C_2
\end{bmatrix}\qquad D_{tot}=D_2D_1$$
\subsection{Mise en parallèle}
\begin{figure}[H]
\centering
\includegraphics[scale=1,page=3]{drwg_0.pdf}
\end{figure}
$$S_{tot}(s)=S_1(s)+S_2(s)$$
$$A_{tot}=\begin{bmatrix}
A_1 & 0\\0 & A_2
\end{bmatrix}\qquad B_{tot}=\begin{bmatrix}
B_1\\B_2
\end{bmatrix}$$
$$C_{tot}=\begin{bmatrix}
C_1 & C_2
\end{bmatrix}\qquad D_{tot}=D_1+D_2$$
\subsubsection{Mise en contre-réaction 1}
\begin{figure}[H]
\centering
\includegraphics[scale=1,page=4]{drwg_0.pdf}
\end{figure}
$$A_{tot}=\begin{bmatrix} A_1 - B_1D_2(I-D_1D_2)^{-1}C_1 & -B_1(C_2-D_2D_1C_2)\\B_2(I-D_1D_2)^{-1}C_1 & A_2-B_2(I-D_1D_2)^{-1}D_1C_2\end{bmatrix}$$

$$B_{tot}=\begin{bmatrix}B_1-B_1D_2ND_1\\B_2ND_1\end{bmatrix}$$

$$C_{tot}=\begin{bmatrix}(I-D_1D_2)^{-1}C_1 & -(I-D_1D_2)^{-1}D_1C_2\end{bmatrix}$$

$$D_{tot}=(I-D_1D_2)^{-1}D_1$$ 

\subsubsection{Mise en contre-réaction 2}
\begin{figure}[H]
\centering
\includegraphics[scale=1,page=5]{drwg_0.pdf}
\end{figure}
$$S_{tot}(s)=\left(I+S_1(s)S_2(s)\right)^{-1}S_1(s)$$
$$A_{tot}=\begin{bmatrix}A_1 & 0\\B_2C_1 & A_2\end{bmatrix}\qquad B_{tot}=\begin{bmatrix}B_1\\B_2D_1\end{bmatrix}$$
$$C_{tot}=\begin{bmatrix}
D_2 C_1 & C_2
\end{bmatrix}\qquad D_{tot}=D_1D_2$$
\subsection{Commandabilité}
$$\boxed{P_c=\begin{bmatrix}B & AB & \cdots & A^{n-1}B\end{bmatrix}}$$
Pour des systèmes monoentrée :
$$\det(P_c)\neq 0\longrightarrow \text{Commandable}$$
Pour des systèmes multi-entrées (généralisation) :
$$\text{rang}(P_c)==n\longrightarrow \text{Commandable}$$
Faire une permutation avec $T$ ne change pas la commandabilité du système. La nouvelle matrice $\tilde{P}_c$ est donnée par $T^{-1}P_c$
\subsection{Observabilité}
$$\boxed{P_0=\begin{bmatrix}C\\CA\\CA^2\\\vdots\\CA^{n-1}\end{bmatrix}}$$
$$\boxed{\text{rang}(P_0)=n\longrightarrow\text{Observable}}$$
En monosortie on peut utiliser $\det(P_0)\neq 0\longrightarrow \text{Observable}$
\subsection{Trajectoire}
\subsection{Système numérique}
Soit un système numérique avec les matrices
$$A_n\qquad B_n\qquad C_n\qquad D_n$$
Et la condition initiale $x_0$. On chercher à trouver les valeurs de $x[0],x[1],x[2],\cdots$
$$x[k]=\textcolor{OrangeRed}{A_n^kx[0]}+\textcolor{RoyalBlue}{A_n^{k-1}B_nu[0]+A_n^{k-2}B_nu[1]+\cdots +B_nu[k-1]}$$
On a la \textcolor{OrangeRed}{contribution de la condition initiale} et un \textcolor{RoyalBlue}{produit de convolution} $u[k]\ast g_x[k]$
\subsubsection{Réponse impulsionnelle}
Si on suppose que la condition initiale est nulle et qu'on excite le signal avec un dirac numérique, alors on a
$$\boxed{x[k]=A_n^{k-1}B_n}$$
\subsection{Système analogique}
Soit un système numérique avec les matrices
$$A_n\qquad B_n\qquad C_n\qquad D_n$$
Et la condition initiale $x_0$
$$\boxed{x(t)=\textcolor{OrangeRed}{e^{At}x_0}+\textcolor{RoyalBlue}{\int_{0}^{t}e^{A(t-\tau)}Bu(\tau)d\tau}}$$
\subsubsection{Exponentielle matricielle (ou matrice de transition)}
$$\boxed{e^{At}=I+At+\frac{(At)^2}{2!}+\frac{(At)^3}{3!}+\cdots}$$
Si $A$ est diagonale, on peut simplifier en écrivant
$$e^{At}=\begin{bmatrix}e^{a_{11}t} & 0 & 0\\0 & e^{a_{22}t} & 0\\0 & 0 & e^{a_{33}t}\end{bmatrix}$$
\paragraph{Calcul par diagonalisation}
Si $A$ est diagonalisable, alors
$$\boxed{e^{At}=Te^{\tilde{A}t}T^{-1}}$$
Ceci permet de simplifier les calculs en utilisant la propriété de l'exponentielle lorsque $\tilde{A}$ est diagonal

\subsection{Forme commandable}
\begin{figure}[H]
\centering
\includegraphics[width=\columnwidth]{img_0.png}
\end{figure}
On obtient donc finalement
$$\boxed{\begin{split}
A&= \begin{bmatrix}0 & 1 & 0\\0 & 0 & 1\\-a_0 & -a_1 & -a_2\end{bmatrix} & B&=\begin{bmatrix}0\\0\\1\end{bmatrix}\\
C&=\begin{bmatrix}b_0 & b_1 & b_2\end{bmatrix} & D&=0
\end{split}}$$
Voir \ref{sec_eq_diff}

\subsection{Modèle échantillonné}
$$\boxed{H(z)=\frac{z-1}{z}Z\left(\mathcal{L}^{-1}\left(\frac{G_a(s)}{s}\right)\Big|_{t=kh}\right)}$$
\subsubsection{Représentation dans l'espace d'état}
$$\boxed{\begin{split}
A_n&=e^{Ah} & B_n&=\int_{0}^{h}e^{A\tau}Bd\tau\\
C_d&=C & D_n &= D
\end{split}}$$



\subsection{Action intégrale sur la commande}
\begin{figure}[H]
\centering
\includegraphics[width=\columnwidth]{drwg_2.pdf}
\end{figure}
$$\hat{A}=\begin{bmatrix}0 & C\\0 & A\end{bmatrix}\qquad \hat{B}=\begin{bmatrix}0\\B\end{bmatrix}$$


\subsection{Placement de pôles}
Il faut que le polynôme caractéristique de la boucle fermée (par exemple $sI-A_{bf}=sI-(A-BK)$) corresponde aux pôles que l'ont souhaite
$$\det(sI-A_{bf})=(s-p_1)(s-p_2)\cdots(s-p_n)$$

\subsection{Retour d'état}
$$A_{bf}=A-BK$$

\subsection{Observateur}
Matrice $A$ de l'observateur (pour le calcul des pôles)
$$A_{obs}=A-HC$$
\begin{center}
\includegraphics[width=\columnwidth]{drwg_5.pdf}
\end{center}
La fonction de transfert du régulateur est

$$G_c(s)=\frac{U(s)}{E(s)}=K\left(sI-A+BK+HC\right)^{-1}H$$


\subsubsection{Filtre}
Si il faut exprimer un filtre, par exemple $\hat{X}_2=\cdots$, il faut commencer par décrire $\hat{X}_1$ en fonction du reste puis résoudre le système (en tout cas dans l'exercice 8.1).\\
Les pôles de l'observateur dans ce cas peuvent se baser sur le dénominateur de la fonction de transfert (si elle est donnée ou calculée).

\subsubsection{Trajectoire}
$$\boxed{\hat{x}[k+1]=Ax[k] + Bu[k] + H\left(y[k] - \hat{y}[k]\right)}$$







\end{document}