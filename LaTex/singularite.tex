\documentclass[resume]{subfiles}

\begin{document}
\section{Calculs sur matrice}

\subsection{Valeurs singulières d'une matrice}

avoir une notion quantitative pas juste booléen (commandabilité, observabilité)

toujours positives et tiré de la plus grande à la plus petite

\subsection{Norme d'une matrice}
$$||M||_2 = max_{x\neq0}(\frac{||Mx||_2}{||x||_2}) = \sigma_{max}$$

le vecteur nul n'est pas pris en compte

\subsection{SVD décomposition en valeurs singulières}
$$M = U\cdot S\cdot V^T$$
M : mXn, U : mXm, S : mXn, $V^T$ : nXn

\begin{itemize}
\item S = $\begin{bmatrix}\sigma_1 & 0& 0\\0&\sigma_2 & 0\\0&0&\sigma_3\\\end{bmatrix}$ @ $\sigma_1 > \sigma_2 > \sigma_3$ 
  \subitem $\sigma_k[M] = \sqrt{eig[M^TM]}$ 
\item V = $\begin{bmatrix}\vec{v_1}|\vec{v_2}|\vec{v_3}\end{bmatrix}$ 
  \subitem $\vec{v_k}= eig[M^TM]$ : vecteur propres de $M^T M$ 
\item U = $\begin{bmatrix}\vec{u_1}|\vec{u_2}|\vec{u_3}\end{bmatrix}$ 
  \subitem $M\vec{v_k} = \sigma_k\vec{u_k}$ => $\vec{u_k}=\frac{M\vec{v_k}}{||M\vec{v_k}||_2}$  
\end{itemize}

\subsubsection{Propriétés SVD}
\begin{itemize}
\item $\sigma_k[M] = \sqrt{eig[M^TM]}$ racine des valeurs propres de $M^TM$ 
\item applicable sur n'importe quelle matrice
\item toujours réel et positif
\item ordre décroissant
\end{itemize}

\end{document}